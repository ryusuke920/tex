\documentclass[dvipdfmx,uplatex]{jsarticle}

\usepackage{marksheet}

\title{共通テスト数学 予想問題}
\author{@ryusuke\_\_h}
\date{2022年1月16日実施予定}

\begin{document}

\maketitle

太郎君と花子ちゃんは、天和・字一色・大四喜・四暗刻が同時に起こる確率について計算しています。\\
以下の空欄に当てはまる数字を記入してください。\\ \\


太郎:「配牌として考えられる組合せは、${}_{136} \mathrm{C}_{14} = \markbox[label]{19}$だね。」\\ \\

花子:「その通り!大四喜は東南西北が3枚ずつでそれぞれ4枚あるから1枚ずつは不要だね!」\\ \\


太郎:「そうだね。じゃあその組み合わせは$\displaystyle 4^{\markbox[label2]{1}} = \markbox[label3]{3}$で簡単に求まるね。」\\ \\

花子:「さらに字一色ってことは、残りの2枚は白發中のどれか2枚だね!」\\ \\

太郎:「麻雀詳しいね。この選び方は$\displaystyle {}_{\markbox[label4]{1}} \mathrm{C}_{\markbox[label5]{1}} \times \markbox[label6]{1} = \markbox[label7]{2}$通りだね。」\\
 \\

花子:「じゃあ求める確率は$\displaystyle \frac{\markbox[label10]{1}}{\markbox[label9]{16}}$なんだね!」\\ \\

太郎:「その通り。計算力すごいね。」\\ \\ \\

\begin{center}
  (問題文は以上です)
\end{center}

\end{document}
