\documentclass[dvipdfmx,uplatex]{jsarticle}
\def\vector#1{\mbox{\boldmath $#1$}}
\usepackage{qexam} % 問題を書く時とかに必要なやつ
\usepackage{setspace} % 行間開けるのに必要なやつ
\usepackage{amsmath} % 数学やるのに必要なやつ
\usepackage{bm} % 太字にするのに必要なやつ
\usepackage{cases} % 連立方程式を書くのに必要なやつ
\usepackage{amssymb}
\usepackage[hiresbb]{graphicx}
\usepackage{ascmac}
\usepackage{siunitx}
\usepackage{float}
\usepackage{tikz}
\usepackage{circuitikz}
\usepackage{url}
\usepackage{braket}
\usepackage[colorlinks=true, bookmarks=true,
bookmarksnumbered = true, bookmarkstype = toc, linkcolor = blue,
urlcolor=blue, citecolor=blue]{hyperref}
\usepackage[version=3]{mhchem}
\makeatletter
 \renewcommand{\theequation}{
   \thesubsection.\arabic{equation}}
  \@addtoreset{equation}{subsection}
\title{夏休み毎日積分 Day9(解答)}
\author{公立はこだて未来大学 システム情報科学部 B2 日置竜輔}
\date{2020年8月9日}
\begin{document}

\maketitle

\begin{itembox}{day 9}
    \begin{center}
        次の定積分を求めよ。\\
        ※今日はガウス積分です。
    \end{center}
\end{itembox}

\begin{description}
    \item [問1] $\displaystyle \iint_D xdxdy \qquad D = \left\{(x, y) | x + y < 2 , x > 0, y > 0 \right\} $
\end{description}

\begin{align*}
    \iint_D xdxdy &= \int_0^2 dx \int_0^{2-x} xdy \\
    &= \int_0^2 \left(x\left[y\right]_0^{2-x}\right)dx = \int_0^2 x\left(2 - x\right) dx = \left[x^2 - \frac{1}{3}x^3\right]_0^2  = \frac{4}{3}
\end{align*}

\begin{description}
    \item [問2] $\displaystyle \iint_D (1 - x - y)dxdy \qquad D = \left\{ (x, y) | x + y \leqq  1 , x \geqq 0, y \geqq 0 \right\}$
\end{description}

\begin{align*}
    \iint_D (1 - x - y)dxdy &= \int_0^1 \left\{\int_0^{1-x} (1 - x - y)dy\right\}dx\\
    &= \int_0^1 \left[-\frac{(1 - x - y)^2}{2}\right]_0^{1-x} dx \\
    &= \int_0^1 \frac{(1-x)^2}{2} dx \\
    &= \left[ - \frac{(1-x)^3}{6}\right]_0^1\\
    &= \frac{1}{6}
\end{align*}

\newpage

\begin{boxnote}
    〜補足〜
    \begin{center}
        再び重積分を行いました。\\
        グラフを想像したら比較的容易に溶ける問題です。\\
        解説でわからなければ自分で調べるか、個別に聞いてください。
    \end{center}
\end{boxnote}

\end{document}
