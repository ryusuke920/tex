\documentclass[dvipdfmx,uplatex]{jsarticle}
\def\vector#1{\mbox{\boldmath $#1$}}
\usepackage{qexam} % 問題を書く時とかに必要なやつ
\usepackage{setspace} % 行間開けるのに必要なやつ
\usepackage{amsmath} % 数学やるのに必要なやつ
\usepackage{bm} % 太字にするのに必要なやつ
\usepackage{cases} % 連立方程式を書くのに必要なやつ
\usepackage{amssymb}
\usepackage[hiresbb]{graphicx}
\usepackage{ascmac}
\usepackage{siunitx}
\usepackage{float}
\usepackage{tikz}
\usepackage{circuitikz}
\usepackage{url}
\usepackage{braket}
\usepackage[colorlinks=true, bookmarks=true,
bookmarksnumbered = true, bookmarkstype = toc, linkcolor = blue,
urlcolor=blue, citecolor=blue]{hyperref}
\usepackage[version=3]{mhchem}
\makeatletter
 \renewcommand{\theequation}{
   \thesubsection.\arabic{equation}}
  \@addtoreset{equation}{subsection}
\title{夏休み毎日積分 Day11(解答)}
\author{公立はこだて未来大学 システム情報科学部 B2 日置竜輔}
\date{2020年8月11日}
\begin{document}

\maketitle

\begin{itembox}{day 11 }
    \begin{center}
        次の定積分を求めよ。\\
        ※今日はいろいろな問題です。
    \end{center}
\end{itembox}

\begin{description}
    \item [問1] $\displaystyle \int_0^\infty \frac{1}{e^x + e^{-x}} dx$
\end{description}

$ \displaystyle e^x = t$とおいて解いていく。

\begin{align*}
    \int \frac{1}{e^x+e^{-x}} dx = \int \frac{e^x}{e^{2x}+1}dx = \frac{1}{1+t^2}dt
    = \arctan{t}
\end{align*}
であるから、

\begin{align*}
    (与式) =  \left[\arctan{t}\right]_1^{\infty} = \frac{\pi}{2} - \frac{\pi}{4} = \frac{\pi}{4}
\end{align*}

\begin{description}
    \item [問2] $\displaystyle \iint_D \frac{1}{1+x+y+xy}  dxdy \qquad D = \{(x,y) | 1 \leqq x \leqq 4, \quad 2 \leqq y \leqq 3\}$
\end{description}

\begin{align*}
    (与式) &= \int_1^4 dx \int_2^3 \frac{1}{(1+x)(1+y)}dy ~=~ \int_1^4dx \left[\frac{\log(1+y)}{1+x}\right]_{y = 2}^{y = 3} \\
    &= \int_1^4 \frac{\log{4} - \log{3}}{1+x}dx = \left[\left( \log{4}-\log{3} \right)\log\left(1+x\right)\right]_1^4 \\
    &= \left(\log{4}-\log{3}\right)\left(\log{5}-\log{2}\right)
\end{align*}

\begin{boxnote}
    \begin{center}
        1問目は$e^x$をかけることで、きれいな形になります。\\
        2問目は因数分解するということが見抜ければ、一発です。\\
        解説でわからなければ自分で調べるか、個別に聞いてください。
    \end{center}
\end{boxnote}

\end{document}
