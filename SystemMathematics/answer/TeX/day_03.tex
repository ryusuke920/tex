\documentclass[dvipdfmx,uplatex]{jsarticle}
\def\vector#1{\mbox{\boldmath $#1$}}
\usepackage{qexam} % 問題を書く時とかに必要なやつ
\usepackage{setspace} % 行間開けるのに必要なやつ
\usepackage{amsmath} % 数学やるのに必要なやつ
\usepackage{bm} % 太字にするのに必要なやつ
\usepackage{cases} % 連立方程式を書くのに必要なやつ
\usepackage{amssymb}
\usepackage[hiresbb]{graphicx}
\usepackage{ascmac}
\usepackage{siunitx}
\usepackage{float}
\usepackage{tikz}
\usepackage{circuitikz}
\usepackage{url}
\usepackage{braket}
\usepackage[colorlinks=true, bookmarks=true,
bookmarksnumbered = true, bookmarkstype = toc, linkcolor = blue,
urlcolor=blue, citecolor=blue]{hyperref}
\usepackage[version=3]{mhchem}
\makeatletter
 \renewcommand{\theequation}{
   \thesubsection.\arabic{equation}}
  \@addtoreset{equation}{subsection}
\title{夏休み毎日積分 Day3(解答)}
\author{公立はこだて未来大学 システム情報科学部 B2 日置竜輔}
\date{2020年8月3日}
\begin{document}

\maketitle

\begin{itembox}[c]{day 3 }
    \begin{center}
        次の積分を求めよ。\\
        ※今日から少しずつレベル上げていきます〜
    \end{center}
\end{itembox}
\begin{description}
    \item [問1] $ \displaystyle \int \frac{3x^3+12x+1}{x^2+4} dx $
\end{description}

とりあえず分子の次数を下げてみましょう。\\
\begin{align*}
  \frac{3x ^ 3 + 12x + 1}{x ^ 2 + 4}
  &=\frac{3x(x ^ 2 + 4)}{x ^ 2 + 4} + \frac{1}{x ^ 2 + 4} = 3x + \frac{1}{x ^ 2 + 4}
  \end{align*}
となるので、
\begin{align*}
    \int \frac{3x ^ 3 + 12x + 1}{x ^ 2 + 4} dx
    &= 3x + \frac{1}{x ^ 2 + 4} dx \\
    &= \int{3x}dx + \int\frac{1}{x ^ 2 + 4} dx
\end{align*}

\begin{equation*}
    \int \frac{1}{x ^ 2 + 4} dx
\end{equation*}
について、
\begin{equation*}
  x = 2 \tan \frac{\theta}{2}
\end{equation*}
と置くと、
\begin{equation*}
    \frac{x}{2}= \tan \frac{\theta}{2}, \quad dx = \cfrac{1}{\cos ^ 2\cfrac{\theta}{2}}
\end{equation*}
であるから、
\begin{align*}
    \int\frac{1}{x ^ 2 + 4}dx
    &=\int \frac{1}{4}\cfrac{1}{\tan ^ 2\cfrac{\theta}{2}+1}\cfrac{1}{\cos ^ 2\cfrac{\theta}{2}} d\theta \\
    &= \int \frac{1}{4}d\theta\\
    &= \frac{1}{4}\theta
\end{align*}

\begin{equation*}
\frac{\theta}{2} = \arctan\frac{x}{2}
\end{equation*}
より、
\begin{equation*}
\frac{1}{4}\theta = \frac{1}{2}\arctan\frac{x}{2}
\end{equation*}

したがって、\\
\begin{align*}
    (与式) = \frac{3}{2}x ^ 2 + \frac{1}{2}\arctan\frac{x}{2} + C \qquad (Cは積分定数とする) \\\\
\end{align*}

\begin{description}
    \item [問2] $ \displaystyle \int_\frac{1}{2}^4 \frac{1}{8x+3} dx $
\end{description}
\begin{equation*}
  \int \frac{1}{f(x)} dx = \log{f(x)}\frac{1}{f'(x)} + C
\end{equation*}
であることを考えると、
\begin{align*}
    \int_\frac{1}{2}^4  \frac{1}{8x+3} dx
    &= \frac{1}{8}\left[\log(8x+3)\right]_\frac{1}{2}^4 \\
    &= \frac{1}{8}\{\log(32+3)-\log(4+3)\} \\
    &= \frac{1}{8}(\log35 - \log7) \\
    &= \frac{1}{8}\log5
\end{align*}

\begin{boxnote}
    〜補足〜
    \begin{center}
        次数下げの問題です。\\
        分子の方が大きい時はこのような工夫をするようにしましょう。 \\
        解説でわからなければ自分で調べるか、個別に聞いてください。
    \end{center}
\end{boxnote}

\end{document}
