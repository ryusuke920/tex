\documentclass[dvipdfmx,uplatex]{jsarticle}
\def\vector#1{\mbox{\boldmath $#1$}}
\usepackage{qexam} % 問題を書く時とかに必要なやつ
\usepackage{setspace} % 行間開けるのに必要なやつ
\usepackage{amsmath} % 数学やるのに必要なやつ
\usepackage{bm} % 太字にするのに必要なやつ
\usepackage{cases} % 連立方程式を書くのに必要なやつ
\usepackage{amssymb}
\usepackage[hiresbb]{graphicx}
\usepackage{ascmac}
\usepackage{siunitx}
\usepackage{float}
\usepackage{tikz}
\usepackage{circuitikz}
\usepackage{url}
\usepackage{braket}
\usepackage[colorlinks=true, bookmarks=true,
bookmarksnumbered = true, bookmarkstype = toc, linkcolor = blue,
urlcolor=blue, citecolor=blue]{hyperref}
\usepackage[version=3]{mhchem}
\makeatletter
 \renewcommand{\theequation}{
   \thesubsection.\arabic{equation}}
  \@addtoreset{equation}{subsection}
\title{夏休み毎日積分 Day6(解答)}
\author{公立はこだて未来大学 システム情報科学部 B2 日置竜輔}
\date{2020年8月6日}
\begin{document}

\maketitle

\begin{itembox}[c]{day 6}
    \begin{center}
        次の定積分を求めよ。\\
        ※今日から本格的に難しくなります。\\
        解けたら十分自信を持っていいと思います。
    \end{center}
\end{itembox}

\begin{description}
    \item [問1] $ \displaystyle \int_0^2 \frac{2x + 1}{\sqrt{x^2 + 4}} dx $
\end{description}

\begin{align*}
    \int_0^2 \frac{2x + 1}{\sqrt{x^2 + 4}} dx &= \int_0^2 \frac{2x}{\sqrt{x^2 + 4}} dx + \int_0^2 \frac{1}{\sqrt{x^2 + 4}} dx
    \intertext{と分けて考える。}
\end{align*}

前半について、$ t = x ^ 2 + 4 $ とおくと、$ dt = 2xdx $ であり、
$x$ は $ 0 \rightarrow 2 $ のとき、$t$ は $ 4 \rightarrow 8 $であるから、

\begin{align*}
    \int_0^2 \frac{2x}{\sqrt{x^2 + 4}} dx &= \int_4^8 \frac{1}{\sqrt{t}} dt \\
    &= \left[2\sqrt{t}\right]_4^8 \\
    &= 4\sqrt{2} - 4
\end{align*}
となる。\\

後半について、

$x = \tan\theta$と置くと、$\displaystyle d\theta = \frac{2}{\cos^2\theta} d\theta$であり、
$x$ は$0 \rightarrow 2$のとき、$\theta$ は$\displaystyle 0 \rightarrow \frac{\pi}{4}$であるから、


\begin{align*}
    \int_0^2 \frac{1}{\sqrt{x^2 + 4}} dx &= \int_0^\frac{\pi}{4} \frac{1}{2\sqrt{{\tan}^2\theta + 1}} \frac{2}{\cos^2\theta} d\theta \\
    &= \int_0^\frac{\pi}{4} \frac{1}{\cos\theta} d\theta \\
    &= \int_0^\frac{\pi}{4} \frac{\cos\theta}{1 - \sin^2\theta} d\theta
\end{align*}

さらにここで、
$t = \sin\theta$と置くと、$dt = \cos\theta d\theta$であり、$ \theta $ は$ \displaystyle 0 \rightarrow \frac{\pi}{4}$のとき、$t$は$\displaystyle 0 \rightarrow \frac{1}{\sqrt{2}}$であるから、\\

\begin{align*}
    \int_0^2 \frac{1}{\sqrt{x^2 + 4}} dx &= \int_0^\frac{1}{\sqrt{2}} \frac{1}{1 - t^2} dt \\
    &= \int_0^\frac{1}{\sqrt{2}} \frac{1}{2}\left(\frac{1}{1 - t} + \frac{1}{1 + t}\right) dt \\
    &= \frac{1}{2} \left[\log\left| \frac{1 + t}{1 - t} \right|\right]_0^\frac{1}{\sqrt{2}} \\
    &= \frac{1}{2}\log\frac{\sqrt{2}+ 1}{\sqrt{2} - 1} \\
    &= \frac{1}{2}\log(\sqrt{2} + 1)^2 \\
    &= \log(\sqrt{2} + 1)
\end{align*}

\newpage

\begin{description}
    \item [問2] $\displaystyle \int_0^2 \left|x^2 - a^2\right| dx$ を$a$を用いて表せ。\\
\end{description}

$x = a$が積分区間 $0 \leqq x \leqq 2$ に含まれているか否かで場合分けをする。\\
($a$) $0 \leq a\leqq 2$ のとき、

\begin{align*}
    (与式) &= \int_0^a \left\{ - (x ^ 2 - a ^ 2)\right\} dx + \int_a^2 (x^2 - a^2) dx \\
    &= \left[-\frac{x^3}{3} + a^2x\right]_0^a + \left[\frac{x^3}{3} - a^2\right]_a^2 \\
    &= \frac{2}{3}a^3 + \left(\frac{8}{3} - 2a^2 + \frac{2}{3}a ^ 3\right) \\
    &= \frac{4}{3}a^3 - 2a^2 + \frac{8}{3}
\end{align*}

($b$)  $a \geqq 2$ のとき、

\begin{align*}
    (与式) &= \int_0^2 \left\{- (x^2 - a^2)\right\} dx \\
    &= \left[ - \frac{x^3}{3} + a^2x\right]_0^2 \\
    &= 2a^2 - \frac{8}{3}
\end{align*}

したがって、($a$), ($b$) より\\
\[
  \begin{cases}
    $ \displaystyle \frac{4}{3}a^3 - 2a^2 + \frac{8}{3} \qquad (0 \leq a \leqq 2のとき)$ \\
    $ \displaystyle 2a^2 - \frac{8}{3} \qquad ( a \geqq 2 のとき)$
  \end{cases}
\]

\begin{boxnote}
    〜補足〜
    \begin{center}
        問1は難問でしたが、問2は単純な絶対値の場合分けの問題です。\\
        中身の正負で場合分けをすれば終わりですね。\\
        解説でわからなければ自分で調べるか、個別に聞いてください。
    \end{center}
\end{boxnote}

\end{document}
