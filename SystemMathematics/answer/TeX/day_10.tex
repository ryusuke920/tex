\documentclass[dvipdfmx,uplatex]{jsarticle}
\def\vector#1{\mbox{\boldmath $#1$}}
\usepackage{qexam} % 問題を書く時とかに必要なやつ
\usepackage{setspace} % 行間開けるのに必要なやつ
\usepackage{amsmath} % 数学やるのに必要なやつ
\usepackage{bm} % 太字にするのに必要なやつ
\usepackage{cases} % 連立方程式を書くのに必要なやつ
\usepackage{amssymb}
\usepackage[hiresbb]{graphicx}
\usepackage{ascmac}
\usepackage{siunitx}
\usepackage{float}
\usepackage{tikz}
\usepackage{circuitikz}
\usepackage{url}
\usepackage{braket}
\usepackage[colorlinks=true, bookmarks=true,
bookmarksnumbered = true, bookmarkstype = toc, linkcolor = blue,
urlcolor=blue, citecolor=blue]{hyperref}
\usepackage[version=3]{mhchem}
\makeatletter
 \renewcommand{\theequation}{
   \thesubsection.\arabic{equation}}
  \@addtoreset{equation}{subsection}
\title{夏休み毎日積分 Day10(解答)}
\author{公立はこだて未来大学 システム情報科学部 B2 日置竜輔}
\date{2020年8月10日}
\begin{document}

\maketitle

\begin{itembox}{day 10}
    \begin{center}
        次の定積分を求めよ。\\
        今日も今日とて重積分の問題です。\\
        飽きてきたと思うので、明日は変わった問題を出題してみます。
    \end{center}
\end{itembox}

\begin{description}
    \item [問1] $ \displaystyle \iint_D \sqrt{y}dxdy \qquad
    D = \left\{(x,y) | \sqrt{\frac{x}{a}} + \sqrt{\frac{y}{b}} \leqq 1, \quad (a > 0, b > 0) \right\}$
\end{description}

\begin{align*}
    &{\displaystyle}\sqrt{\frac{x}{a}} + \sqrt{\frac{y}{b}} ~=~ 1~のとき~~x ~=~a\left(1~-~\sqrt{\frac{y}{b}}\right)^2 から、\\
    &D = \left\{(x,y)~:~0 ~\leqq~ x ~\leqq~ a\left(1 ~-~\sqrt{\frac{y}{b}}\right)^2,~0 ~\leqq~y~\leqq~b\right\}となるので、\\
\end{align*}

\begin{align*}
    &\iint_D \sqrt{y}dxdy ~=~ \int_0^b\left(\int_0^a\left(1 - \sqrt{\frac{y}{b}}\right)\sqrt{y}dx\right)~dy
    = \int_0^b \sqrt{y}\left[x\right]_{x=0}^{x=a\left(1-\sqrt{\frac{y}{b}}\right)^2}dy \\
    &= a\int_0^b \sqrt{y}\left(1-\sqrt{\frac{y}{b}}\right)^2dy = a\int_0^b\left(\sqrt{y}-\frac{2y}{\sqrt{b}}+\frac{y^{\frac{3}{2}}}{b}\right)dy \\
    &= a\left[\frac{2}{3}y^{\frac{3}{2}} - \frac{2}{\sqrt{b}}frac{y^2}{2} + \frac{1}{b}\frac{2}{5}y^{\frac{5}{2}}\right]_0^b\\
    &= \frac{1}{15}ab^{\frac{3}{2}}
\end{align*}

\begin{description}
    \item [問2] $\displaystyle \int_0^{\frac{\pi}{2}} dx \int_0^{\frac{\pi}{2}} \sin(x + y)dy $
\end{description}

\begin{align*}
    \int_0^{\frac{\pi}{2}} \sin(x+y) dy &= \left[-\cos(x+y)\right]_{y = 0}^{y = \frac{\pi}{2}} \\
    &= \left\{-\cos(x + \frac{\pi}{2})\right\} - \left(-\cos(x)\right) \\
\end{align*}

ここで、$ \displaystyle \cos\left(x + \frac{\pi}{2}\right) = -\sin{x}$を用いると、

\begin{align*}
    \left\{-\cos(x + \frac{\pi}{2})\right\} - \left\{-\cos(x)\right\}  = \sin{x} + \cos{x}\\
\end{align*}

となるので \\

\begin{align*}
    \int_0^{\frac{\pi}{2}} (\sin{x} + \cos{x}) dy = \left[- \cos(x+y)\right]_{y = 0}^{y = \frac{\pi}{2}}
    = (0 + 1)-(-1+0) ~=~ 2
\end{align*}

\begin{boxnote}
    \begin{center}
        解説でわからなければ自分で調べるか、個別に聞いてください。
    \end{center}
\end{boxnote}

\end{document}
