\documentclass[dvipdfmx,uplatex]{jsarticle}
\def\vector#1{\mbox{\boldmath $#1$}}
\usepackage{qexam} % 問題を書く時とかに必要なやつ
\usepackage{setspace} % 行間開けるのに必要なやつ
\usepackage{amsmath} % 数学やるのに必要なやつ
\usepackage{bm} % 太字にするのに必要なやつ
\usepackage{cases} % 連立方程式を書くのに必要なやつ
\usepackage{amssymb}
\usepackage[hiresbb]{graphicx}
\usepackage{ascmac}
\usepackage{siunitx}
\usepackage{float}
\usepackage{tikz}
\usepackage{circuitikz}
\usepackage{url}
\usepackage{braket}
\usepackage[colorlinks=true, bookmarks=true,
bookmarksnumbered = true, bookmarkstype = toc, linkcolor = blue,
urlcolor=blue, citecolor=blue]{hyperref}
\usepackage[version=3]{mhchem}
\makeatletter
 \renewcommand{\theequation}{
   \thesubsection.\arabic{equation}}
  \@addtoreset{equation}{subsection}
\title{夏休み毎日積分 Day10(問題)}
\author{公立はこだて未来大学 システム情報科学部 B2 日置竜輔}
\date{2020年8月10日}
\begin{document}

\maketitle

\begin{itembox}[c]{day 10}
    \begin{center}
        次の定積分を求めよ。\\
        今日も今日とて重積分の問題です。\\
        飽きてきたと思うので、明日は変わった問題を出題してみます。
    \end{center}
\end{itembox}

\begin{description}
    \item [問1] $ \displaystyle \iint_D \sqrt{y}dxdy \qquad
    D = \left\{(x,y) | \sqrt{\frac{x}{a}} + \sqrt{\frac{y}{b}} \leqq 1, \quad (a > 0, b > 0) \right\}$
\end{description}

\begin{description}
    \item [問2] $\displaystyle \int_0^{\frac{\pi}{2}} dx \int_0^{\frac{\pi}{2}} \sin(x + y)dy $
\end{description}
\end{document}
