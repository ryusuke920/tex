\documentclass[dvipdfmx,uplatex]{jsarticle}
\def\vector#1{\mbox{\boldmath $#1$}}
\usepackage{qexam} % 問題を書く時とかに必要なやつ
\usepackage{setspace} % 行間開けるのに必要なやつ
\usepackage{amsmath} % 数学やるのに必要なやつ
\usepackage{bm} % 太字にするのに必要なやつ
\usepackage{cases} % 連立方程式を書くのに必要なやつ
\usepackage{amssymb}
\usepackage[hiresbb]{graphicx}
\usepackage{ascmac}
\usepackage{siunitx}
\usepackage{float}
\usepackage{tikz}
\usepackage{circuitikz}
\usepackage{url}
\usepackage{braket}
\usepackage[colorlinks=true, bookmarks=true,
bookmarksnumbered = true, bookmarkstype = toc, linkcolor = blue,
urlcolor=blue, citecolor=blue]{hyperref}
\usepackage[version=3]{mhchem}
\makeatletter
 \renewcommand{\theequation}{
   \thesubsection.\arabic{equation}}
  \@addtoreset{equation}{subsection}
\title{夏休み毎日積分 Day7(問題)}
\author{公立はこだて未来大学 システム情報科学部 B2 日置竜輔}
\date{2020年8月7日}
\begin{document}

\maketitle

\begin{itembox}[c]{day 7}
    \begin{center}
        次の定積分を求めよ。\\
        ※今日はガウス積分です。
    \end{center}
\end{itembox}

\begin{description}
    \item [問1] $\displaystyle \int_{-\infty}^{\infty} e^{-ax^2} dx$
\end{description}

\begin{description}
    \item [問2] $\displaystyle \int_{-\infty}^{\infty} e^{-a(x-b)^2} dx$
\end{description}

\begin{description}
    \item [問3] $\displaystyle \int_0^{\infty} e^{-ax^2} dx$
\end{description}

\end{document}
