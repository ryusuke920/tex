\documentclass[dvipdfmx,uplatex]{jsarticle}
\def\vector#1{\mbox{\boldmath $#1$}}
\usepackage{qexam} % 問題を書く時とかに必要なやつ
\usepackage{setspace} % 行間開けるのに必要なやつ
\usepackage{amsmath} % 数学やるのに必要なやつ
\usepackage{bm} % 太字にするのに必要なやつ
\usepackage{cases} % 連立方程式を書くのに必要なやつ
\usepackage{amssymb}
\usepackage[hiresbb]{graphicx}
\usepackage{ascmac}
\usepackage{siunitx}
\usepackage{float}
\usepackage{tikz}
\usepackage{circuitikz}
\usepackage{url}
\usepackage{braket}
\usepackage[colorlinks=true, bookmarks=true,
bookmarksnumbered = true, bookmarkstype = toc, linkcolor = blue,
urlcolor=blue, citecolor=blue]{hyperref}
\usepackage[version=3]{mhchem}
\makeatletter
 \renewcommand{\theequation}{
   \thesubsection.\arabic{equation}}
  \@addtoreset{equation}{subsection}
\title{春休み毎日微分方程式 Day 9(解答)}
\author{公立はこだて未来大学 システム情報科学部 B2 日置竜輔}
\date{\today}

\begin{document}
\begin{spacing}{1.6}
\maketitle

\question{問1}
以下の連立微分方程式を解け。
\begin{qparts}
  \qpart
  \begin{numcases}
    {}
    \frac{dy}{dt} = 8x + 2y \nonumber \\
    \frac{dx}{dt} = -6x + y \nonumber
  \end{numcases}

  \subsection*{解答}

  この問題は行列を用いると、
  \[
  \frac{d}{dt}
  \left(
  \begin{array}{ccc}
    x \\
    y
  \end{array}
  \right)
  = \left(
    \begin{array}{ccc}
      8 & 2 \\
      -6 & 1 \\
    \end{array}
  \right)
  \left(
  \begin{array}{ccc}
    x \\
    y
  \end{array}
  \right)
  \]
  と表すことができるので、
  \begin{equation*}
    A = \left(
        \begin{array}{ccc}
          8 & 2 \\
          -6 & 1
        \end{array}
      \right)
  \end{equation*}
  とおくと、
  \[
  \frac{d}{dt}
  \left(
  \begin{array}{ccc}
    x \\
    y
  \end{array}
  \right)
  = A
  \left(
  \begin{array}{ccc}
    x \\
    y
  \end{array}
  \right)
  \]
  となる。この$A$について、固有値を求めると、
  \begin{eqnarray*}
    |A - \lambda I| & = & \left|
        \begin{array}{ccc}
          8 - \lambda & 2 \\
          -6 & 1 - \lambda
        \end{array}
      \right| \\
       & = & (\lambda - 1)(\lambda - 8) + 12 \\
       & = & \lambda ^ 2 - 9 \lambda + 20 \\
       & = & (\lambda - 4)(\lambda - 5) \\
       & = & 0
  \end{eqnarray*}
  となるので、
  \begin{equation*}
    \lambda = 4, 5
  \end{equation*}
  となる。次に、対応する固有値ごとに固有ベクトルを求める。\\

  \subsubsection*{(ⅰ) $ \lambda = 4 $ の時、}

  \begin{eqnarray*}
    A - 4I & = &
    \left(
    \begin{array}{ccc}
      4 & 2 \\
      -6 & -3
    \end{array}
    \right)
    \rightarrow
    \left(
    \begin{array}{ccc}
      4 & 2 \\
      2 & 1
    \end{array}
    \right)
    \rightarrow
    \left(
    \begin{array}{ccc}
      2 & 1 \\
      0 & 0
    \end{array}
    \right)
  \end{eqnarray*}
  となる。この時、
  \begin{eqnarray*}
    (A - 4I) \left(
    \begin{array}{ccc}
      x \\
      y
    \end{array}
    \right)
    & = & 0 \\
    \Leftrightarrow
    2x + y & = & 0
  \end{eqnarray*}
  が得られ、この方程式の解は任意定数$k$を用いて、
  \begin{equation*}
    \left(
    \begin{array}{ccc}
      x \\
      y
    \end{array}
    \right)
    = k
    \left(
    \begin{array}{ccc}
      1 \\
      -2
    \end{array}
    \right)
  \end{equation*}
  と表すことができるので、固有値ベクトル$\overrightarrow{p_1}$は、
  \begin{equation}
    \overrightarrow{p_1} =
    \left(
    \begin{array}{ccc}
      1 \\
      -2
    \end{array}
    \right)
  \end{equation}
  となる。\\

  同様に、固有値ベクトルをもう1つ求める。

  \subsubsection*{(ⅱ) $ \lambda = 5 $ の時、}

  \begin{eqnarray*}
    A - 5I & = &
    \left(
    \begin{array}{ccc}
      3 & 2 \\
      -6 & -4
    \end{array}
    \right)
    \rightarrow
    \left(
    \begin{array}{ccc}
      3 & 2 \\
      3 & 2
    \end{array}
    \right)
    \rightarrow
    \left(
    \begin{array}{ccc}
      3 & 2 \\
      0 & 0
    \end{array}
    \right)
  \end{eqnarray*}
  となる。この時、
  \begin{eqnarray*}
    (A - 5I) \left(
    \begin{array}{ccc}
      x \\
      y
    \end{array}
    \right)
    & = & 0 \\
    \Leftrightarrow
    3x + 2y & = & 0
  \end{eqnarray*}
  が得られ、この方程式の解は任意定数$k$を用いて、
  \begin{equation*}
    \left(
    \begin{array}{ccc}
      x \\
      y
    \end{array}
    \right)
    = k
    \left(
    \begin{array}{ccc}
      2 \\
      -3
    \end{array}
    \right)
  \end{equation*}
  と表すことができるので、固有値ベクトル$\overrightarrow{p_2}$は、
  \begin{equation}
    \overrightarrow{p_2} =
    \left(
    \begin{array}{ccc}
      2 \\
      -3
    \end{array}
    \right)
  \end{equation}
  となる。

  \newpage

  したがって、(0.0.1),(0.0.2)より、行列$P$を
  \begin{equation*}
    P = (\overrightarrow{p_1} \quad \overrightarrow{p_2}) =
    \left(
    \begin{array}{ccc}
      1 & 2 \\
      -2 & -3
    \end{array}
    \right)
  \end{equation*}
  とすることができ、
  \begin{equation}
    P^{-1}AP =
    \left(
    \begin{array}{ccc}
      4 & 0 \\
      0 & 5
    \end{array}
    \right)
    = D
  \end{equation}
  と対角化することができる。\\\\

  次に $ P^{-1}AP $ つまり、 $ A = PDP^{-1} $より、
  \begin{equation*}
    \frac{d \overrightarrow{x}}{dt} = PDP^{-1} \overrightarrow x
  \end{equation*}
  と変形できる。さらに、$ P^{-1} $ を左辺に掛けることにより、
  \begin{equation*}
    P^{-1}\frac{d \overrightarrow{x}}{dt} = P^{-1}PDP^{-1}\overrightarrow x
  \end{equation*}
  となる。ここで、$P^{-1}$は定数なので、
  \begin{equation*}
    \frac{d}{dt}P^{-1}\overrightarrow x = DP^{-1}\overrightarrow x
  \end{equation*}
  と変形でき、さらに$\overrightarrow y = P^{-1}\overrightarrow x$とおくことで、
  \begin{eqnarray*}
    \frac{d}{dt}\overrightarrow y & = & D \overrightarrow y \\
  \end{eqnarray*}
  \newpage

  同様に、
  \begin{eqnarray*}
    \frac{d}{dt}\overrightarrow x & = & D \overrightarrow x \\
    \frac{d}{dt}
    \left(
    \begin{array}{ccc}
    X \\
    Y
    \end{array}
    \right)
    & = &
    \left(
    \begin{array}{ccc}
    4 & 0 \\
    0 & 5
    \end{array}
    \right)
    \left(
    \begin{array}{ccc}
    X \\
    Y
    \end{array}
    \right)
  \end{eqnarray*}
  と変形できる。\\\\
  したがって、任意定数$C_1, C_2$を用いて、
  \begin{eqnarray*}
    \frac{dX}{dt} = k_1X & \Rightarrow & X  = C_1e ^{4t}\\
    \frac{dY}{dt} = k_2X & \Rightarrow & Y  = C_2e ^{5t}
  \end{eqnarray*}
  と変形できる。{\bf (変数分離系を用いて式変形をする。)}\\
  また、$\overrightarrow y = P^{-1} \overrightarrow x$つまり、$\overrightarrow x = P \overrightarrow y$なので、
  \begin{eqnarray*}
    \overrightarrow x & = & P \overrightarrow y \\
    & = &
    \left(
    \begin{array}{ccc}
    1 & 2 \\
    -2 & -3
    \end{array}
    \right)
    \left(
    \begin{array}{ccc}
    X \\
    Y
    \end{array}
    \right) \\
    & = &
    \left(
    \begin{array}{ccc}
    1 & 2 \\
    -2 & -3
    \end{array}
    \right)
    \left(
    \begin{array}{ccc}
    C_1 e ^ {4t} \\
    C_2 e ^ {5t}
    \end{array}
    \right) \\
    & = &
    \left(
    \begin{array}{ccc}
    C_1 e ^ {4t} + 2 C_2 e ^ {5t} \\
    -2C_1 e ^ {4t} - 3 C_2 e ^ {5t}
    \end{array}
    \right) \\
    & = &
    \left(
    \begin{array}{ccc}
    x \\
    y
    \end{array}
    \right) \\
  \end{eqnarray*}
  となるので、求める一般解は

  \begin{numcases}
    {}
    x = C_1 e ^ {4t} + 2 C_2 e ^ {5t} \nonumber \\
    y = -2C_1 e ^ {4t} - 3 C_2 e ^ {5t} \nonumber
  \end{numcases}
  \end{qparts}
\end{spacing}
\end{document}
