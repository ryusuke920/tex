\documentclass[dvipdfmx,uplatex]{jsarticle}
\def\vector#1{\mbox{\boldmath $#1$}}
\usepackage{qexam} % 問題を書く時とかに必要なやつ
\usepackage{setspace} % 行間開けるのに必要なやつ
\usepackage{amsmath} % 数学やるのに必要なやつ
\usepackage{bm} % 太字にするのに必要なやつ
\usepackage{amssymb}
\usepackage[hiresbb]{graphicx}
\usepackage{ascmac}
\usepackage{siunitx}
\usepackage{float}
\usepackage{tikz}
\usepackage{circuitikz}
\usepackage{url}
\usepackage{braket}
\usepackage[colorlinks=true, bookmarks=true,
bookmarksnumbered = true, bookmarkstype = toc, linkcolor = blue,
urlcolor=blue, citecolor=blue]{hyperref}
\usepackage[version=3]{mhchem}
\makeatletter
 \renewcommand{\theequation}{
   \thesubsection.\arabic{equation}}
  \@addtoreset{equation}{subsection}
\title{春休み毎日微分方程式 Day 6(解答)}
\author{公立はこだて未来大学 システム情報科学部 B2 日置竜輔}
\date{\today}

\begin{document}
\begin{spacing}{1.6}
\maketitle

\question{問1}
以下の証明をせよ。 ($ \displaystyle D = \frac{d}{dx} $ を表す。)
\begin{qparts}
    \qpart {\bm 微分演算子} と {\bm 定数変化法}を用いて、
    $ \displaystyle \frac{dy}{dx} + ay = f(x) から $
    \begin{equation}
      \frac{1}{D + a}f(x) = e ^ {-ax} \left(\int e ^ {ax} f(x) dx \right )
    \end{equation}を導出せよ。 \\
    定数変化法を用いると、
    \begin{equation}
      y = e ^ {-ax} \left( \int e^{ax}dx \right)
    \end{equation}
    と表すことができる。\\
    また、$ \displaystyle \frac{dy}{dx} + ay = f(x) から $は、微分演算子を用いることで、\\
    \begin{eqnarray}
      Dy + ay =  & f(x) & \nonumber \\
      (D + a)y = & f(x) & \nonumber \\
      y = \frac{1}{D + a} & f(x) &
      \end{eqnarray}
      と変形することができる。
    したがって、(0.0.2),(0.0.3)の2式より、\\
    \begin{equation*}
      \frac{1}{D + a}f(x) = e ^ {-ax} \left(\int e ^ {ax} f(x) dx \right )
    \end{equation*}
    となる。\\
  \end{qparts}
  この公式は導出出来ないと、後期の微分方程式でも頻繁に使用するのでできるようにしておきましょう。\\
  \begin{itembox}{微分演算子法の超基本事項(おさらい)}
    {\rm I} \quad $ D $ を重ね掛けして複数回の微分を表現可能 \\
    \begin{equation*}
      D^2 = \frac{d^2}{dx},\qquad D^3 = \frac{d ^ 3}{dx}, \qquad D ^ n = \frac{d^n}{dx}
    \end{equation*}
    {\rm II}\quad 定数は $ D $ の前に出せる。\\
    \begin{equation*}
      D^naf(x) = aD^nf(x)
    \end{equation*}
    {\rm III}\quad $ D $は分配・結合ができる。\\
    \begin{equation*}
      (D ^ m + D ^ n)f(x) = D^mf(x) + D^nf(x)
    \end{equation*}
  \end{itembox}
\end{spacing}

\end{document}
