\documentclass[dvipdfmx,uplatex]{jsarticle}
\def\vector#1{\mbox{\boldmath $#1$}}
\usepackage{qexam} % 問題を書く時とかに必要なやつ
\usepackage{setspace} % 行間開けるのに必要なやつ
\usepackage{amsmath} % 数学やるのに必要なやつ
\usepackage{bm} % 太字にするのに必要なやつ
\usepackage{cases} % 連立方程式を書くのに必要なやつ
\usepackage{amssymb}
\usepackage[hiresbb]{graphicx}
\usepackage{ascmac}
\usepackage{siunitx}
\usepackage{float}
\usepackage{tikz}
\usepackage{circuitikz}
\usepackage{url}
\usepackage{braket}
\usepackage[colorlinks=true, bookmarks=true,
bookmarksnumbered = true, bookmarkstype = toc, linkcolor = blue,
urlcolor=blue, citecolor=blue]{hyperref}
\usepackage[version=3]{mhchem}
\makeatletter
 \renewcommand{\theequation}{
   \thesubsection.\arabic{equation}}
  \@addtoreset{equation}{subsection}
\title{春休み毎日微分方程式 Day 10(解答)}
\author{公立はこだて未来大学 システム情報科学部 B2 日置竜輔}
\date{\today}

\begin{document}
\begin{spacing}{1.6}
\maketitle

\question{問1}
以下の連立微分方程式を解け。\\
\begin{qparts}
  \qpart
  \begin{equation*}
    \frac{dy}{dx} + 3y = e ^ x
  \end{equation*}

  \subsection*{解答}

  微分方程式を演算子法を用いて表すと、
  \begin{equation*}
    Dy + 3y = (D + 3)y = e ^ x
  \end{equation*}
  となるので、
  \begin{equation*}
    y = \frac{1}{D + 3}e ^ x
  \end{equation*}
  になるので、
  \begin{eqnarray*}
    \frac{1}{D + 3}e ^ x & = & e ^ {-3x} \int e ^ {3x} e ^ x dx \\
    & = & e ^ {-3x} \int e ^ {3x} e ^ x dx \\
    & = & e ^ {-3x} \int e ^ {4x} dx \\
    & = & e ^ {-3x} \frac{1}{4}e ^ {4x} \\
    & = & \frac{1}{4}e ^ x + C \qquad (Cは任意定数)
  \end{eqnarray*}
  となる。

  \newpage

  \qpart
  \begin{equation*}
    \frac{d^2y}{dx^2} - 7 \frac{dy}{dx} + 12y = 3x
  \end{equation*}

  \subsection*{解答}

  微分方程式を演算子法を用いて表すと、
  \begin{eqnarray*}
    D ^ 2y - 7Dy + 12y & = & (D ^ 2 - 7D + 12)y \\
    & = & (D - 3)(D - 4)y \\
    & = & 3x \\
    \Leftrightarrow
    y & = & \frac{1}{(D - 3)(D - 4)}3x
   \end{eqnarray*}
   となる。\\

   よって、
   \begin{eqnarray*}
     \frac{1}{(D - 3)(D - 4)}3x & = & \frac{1}{D - 3} \left(\frac{1}{D - 4}3x\right) \\
     & = & \frac{1}{D - 3}e ^ {-(-4)x}\left(\int e ^ {-4x}3xdx\right) \\
     & = & \frac{1}{D - 3}e ^ {4x} \left(-\frac{1}{4}e ^ {-4x}3x - \frac{3}{16}e ^ {-4x}\right) \\
     & = & \frac{1}{D - 3} \left(-\frac{1}{16}e ^ {4x}e ^ {-4x} (12x + 3)\right) \\
     & = & -\frac{3}{16} \frac{1}{D + 3}(4x + 1) \\
     & = & -\frac{3}{16} e ^{-(-3)x} \left(\int e ^ {-3x}(4x + 1) dx \right) \\
     & = & -\frac{3}{16}e ^ {3x} \left(-\frac{1}{3}e ^ {-3x}(4x + 1) - \frac{1}{9}e ^ {-3x} 4 \right) \\
     & = & \frac{3}{16} \frac{1}{9}e ^ {-3x}e ^ {-3x}(12x + 7) \\
     & = & \frac{1}{48}(12x + 7) \\
     & = & \frac{1}{4}x + \frac{7}{48}
   \end{eqnarray*}
   となるので、一般解は
   \begin{equation*}
     y = \frac{1}{4}x + \frac{7}{48}
   \end{equation*}
  \end{qparts}
  \end{spacing}
  \begin{shadebox}
    後期の授業では$D$を用いた方程式を解く際には分母を部分分数分解すると習いましたが、解説のように1つ1つ処理して行っても答えを求めることができるので覚えておきましょう。
  \end{shadebox}
\end{document}
