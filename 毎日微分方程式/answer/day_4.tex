\documentclass[dvipdfmx,uplatex]{jsarticle}
\def\vector#1{\mbox{\boldmath $#1$}}
\usepackage{qexam} % 問題を書く時とかに必要なやつ
\usepackage{setspace} % 行間開けるのに必要なやつ
\usepackage{amsmath} % 数学やるのに必要なやつ
\usepackage{bm} % 太字にするのに必要なやつ
\usepackage{amssymb}
\usepackage[hiresbb]{graphicx}
\usepackage{ascmac}
\usepackage{siunitx}
\usepackage{float}
\usepackage{tikz}
\usepackage{circuitikz}
\usepackage{url}
\usepackage{braket}
\usepackage[colorlinks=true, bookmarks=true,
bookmarksnumbered = true, bookmarkstype = toc, linkcolor = blue,
urlcolor=blue, citecolor=blue]{hyperref}
\usepackage[version=3]{mhchem}
\makeatletter
 \renewcommand{\theequation}{
   \thesubsection.\arabic{equation}}
  \@addtoreset{equation}{subsection}
\title{春休み毎日微分方程式 Day 4(解答)}
\author{公立はこだて未来大学 システム情報科学部 B2 日置竜輔}
\date{\today}

\begin{document}
\begin{spacing}{1.6}
\maketitle

\question{問1}
次の一階線形微分方程式を解け。
\begin{qparts}
    \qpart $ \displaystyle \frac{dy}{dx} - y = e ^ {2x} $ \\
    一階線形微分方程式は一般的に次のように表される。\\
    \begin{itembox}{公式}
       \begin{equation}
         \frac{dy}{dx} + p(x)y = q(x) \nonumber
       \end{equation}
    \end{itembox}
     この時、解である $ y $は、\\
     \begin{equation}
       P(x) = \int p(x) dx \nonumber
     \end{equation}
     \begin{equation}
       \mu (x) = e ^ {P(x)} \nonumber
     \end{equation}
     として、 $ \displaystyle y = \frac{1}{\mu} \int \mu q(x) dx $ で求められる。\\
     この問題の場合だと、 $ p(x) =-1 $、$q(x) = e ^ {2x} $ となるので、 $ P(x) = - x $ となる。 \\
     したがって、$ \mu (x) = e ^ {-x} $ となり、 $ y $ は次のように求められる。\\
     \begin{eqnarray*}
       y & = & \frac{1}{\mu} \int \mu q(x) dx \\
       & = & \frac{1}{e ^ {-x}} \int e ^ {-x} e ^ {2x} dx \\
       & = & \frac{1}{e ^ {-x}} \int e ^ x dx \\
       & = & e ^ x (e ^ x + C) \\
       & = & e ^ {2x} + Ce ^ x \qquad (Cは任意定数) \\
     \end{eqnarray*}

     \qpart {\rm I} で解いた微分方程式が解を満たすことを示せ。
     {\rm I} で求めた $ y $ を実際に代入すると、 \\
     \begin{eqnarray*}
       \frac{dy}{dx} - y & = & {(e ^ {2x} + Ce ^ x)\prime} - (e ^ {2x} + Ce ^ x) \\
       & = & 2 e ^ {2x} + Ce ^ x - e ^ {2x} - Ce ^ x \\
       & = & e ^ {2x}
     \end{eqnarray*}
     したがって、この解は成り立つ。
\end{qparts}

\end{spacing}
\end{document}
