\documentclass[dvipdfmx,uplatex]{jsarticle}
\def\vector#1{\mbox{\boldmath $#1$}}
\usepackage{qexam} % 問題を書く時とかに必要なやつ
\usepackage{setspace} % 行間開けるのに必要なやつ
\usepackage{amsmath} % 数学やるのに必要なやつ
\usepackage{bm} % 太字にするのに必要なやつ
\usepackage{amssymb}
\usepackage[hiresbb]{graphicx}
\usepackage{ascmac}
\usepackage{siunitx}
\usepackage{float}
\usepackage{tikz}
\usepackage{circuitikz}
\usepackage{url}
\usepackage{braket}
\usepackage[colorlinks=true, bookmarks=true,
bookmarksnumbered = true, bookmarkstype = toc, linkcolor = blue,
urlcolor=blue, citecolor=blue]{hyperref}
\usepackage[version=3]{mhchem}
\makeatletter
 \renewcommand{\theequation}{
   \thesubsection.\arabic{equation}}
  \@addtoreset{equation}{subsection}
\title{春休み毎日微分方程式 Day 3(解答)}
\author{公立はこだて未来大学 システム情報科学部 B2 日置竜輔}
\date{\today}

\begin{document}
\begin{spacing}{1.6}
\maketitle

\question{問1}
以下の微分方程式を解け。
\begin{qparts}
    \qpart $ \displaystyle \frac{dy}{dx} = \frac{x - y}{x + y} $ \\

    右辺の分子分母を $ x $ で割ると、\\
    \begin{equation}
      \frac{dy}{dx} = \frac{1 - \cfrac{y}{x}}{1 + \cfrac{y}{x}} \nonumber
    \end{equation}
    この形は同次形であるから、$ \displaystyle u = \frac{y}{x} $ の変数変換を行うと、
    \begin{equation}
      \frac{dy}{dx} = \frac{1 - u}{1 + u}
    \end{equation}
    となり、また $ \displaystyle u = \frac{y}{x} $ より $ y = ux $ が導かれるので、\\
    これらの両辺を $ x $ で微分すると、\\
    \begin{equation}
      \frac{dy}{dx} = \frac{dy}{dx}x + u
    \end{equation}
    (0.0.1), (0.0.2)より、
    \begin{equation}
      \frac{1 - u}{1 + u} = \frac{dy}{dx}x + u \nonumber
    \end{equation}
    が得られる。\\
    これを $ \displaystyle \frac{du}{dx} $ について整理すると、\\
    \begin{equation}
      \frac{du}{dx} = - \frac{u ^ 2 + 2u - 1}{u + 1}\frac{1}{x} \nonumber
    \end{equation}
    これは{\bf 変数分離系}の形であるから、$ u ^ 2 + 2u - 1 \neq 0 $ を仮定して、\\
    $ u $ を左辺に、$ x $ を右辺にまとめると、
    \begin{equation}
      \frac{u + 1}{u ^ 2 + 2u - 1} \frac{du}{dx} = -\frac{1}{x} \nonumber
    \end{equation}
    この両辺を $ x $ で積分すると、
    \begin{equation}
      \int \frac{u + 1}{u ^ 2 + 2u - 1} du = - \int \frac{1}{x}dx
    \end{equation}
    \begin{itembox}{公式}
       \begin{equation}
         \int \frac{f{\prime}(x)}{f(x)} dx = \log |f(x)| + C \qquad (Cは任意定数) \nonumber
       \end{equation}
    \end{itembox}
    (0.0.3)の左辺について、上記の公式を用いると、
    \begin{eqnarray}
      \log |u ^ 2 + 2u - 1| & = & -2 \log |x| + C \nonumber \\
      \log |u ^ 2 + 2u - 1| + 2 \log |x| & = & C \nonumber \\
      \log |u ^ 2 + 2u - 1| + \log |x| ^ 2 & = & C \nonumber \\
      \log |(u ^ 2 + 2u - 1)x ^ 2| & = & C \nonumber \\
      |(u ^ 2 + 2u - 1)x ^ 2| & = & \mathrm{e} ^ C \nonumber \\
      (u ^ 2 + 2u - 1)x ^ 2 & = & \pm \mathrm{e} ^ C \nonumber \\
      (u ^ 2 + 2u - 1)x ^ 2 & = & D \nonumber \qquad (D \neq 0) \quad (C, D は任意定数)
    \end{eqnarray}
    ここで、$ u ^ 2 + 2u - 1 = 0 $ を仮定すると、\\
    \begin{equation}
      u = -1 \pm \sqrt{2} \qquad (定数関数)
    \end{equation}
    この時の式は、
    \begin{equation}
      \frac{du}{dx} = 0 \nonumber
    \end{equation}
    であり、これは(0.0.4)の解も満たしている。\\
    したがって、$ D \neq 0 $ はではなく、全ての $ D $ において成り立つことになったので、
    \begin{equation}
      (u ^ 2 + 2u - 1)x ^ 2 = C{\prime} \qquad (C{\prime}は任意定数) \nonumber
    \end{equation}
    したがって、 $ \displaystyle u = \frac{y}{x} $ を代入すると、求める解は
    \begin{equation}
      (y ^ 2 + 2xy - x ^ 2) = C{\prime} \nonumber
    \end{equation}
\end{qparts}
\end{spacing}

\end{document}
