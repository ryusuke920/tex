\documentclass[dvipdfmx,uplatex]{jsarticle}
\def\vector#1{\mbox{\boldmath $#1$}}
\usepackage{qexam} % 問題を書く時とかに必要なやつ
\usepackage{setspace} % 行間開けるのに必要なやつ
\usepackage{amsmath} % 数学やるのに必要なやつ
\usepackage{bm} % 太字にするのに必要なやつ
\usepackage{amssymb}
\usepackage[hiresbb]{graphicx}
\usepackage{ascmac}
\usepackage{siunitx}
\usepackage{float}
\usepackage{tikz}
\usepackage{circuitikz}
\usepackage{url}
\usepackage{braket}
\usepackage[colorlinks=true, bookmarks=true,
bookmarksnumbered = true, bookmarkstype = toc, linkcolor = blue,
urlcolor=blue, citecolor=blue]{hyperref}
\usepackage[version=3]{mhchem}
\makeatletter
 \renewcommand{\theequation}{
   \thesubsection.\arabic{equation}}
  \@addtoreset{equation}{subsection}
\title{春休み毎日微分方程式 Day 7(解答)}
\author{公立はこだて未来大学 システム情報科学部 B2 日置竜輔}
\date{\today}

\begin{document}
\begin{spacing}{1.6}
\maketitle

\question{問1}
次の微分方程式を計算せよ。
\begin{qparts}
    \qpart $ \displaystyle \frac{dy}{dx} = 1 + y ^ 2 $ \\
    これは {\bf 変数分離系}の形であるから、\\
    \begin{equation*}
      \frac{1}{1 + y ^ 2} \frac{dy}{dx} = 1
    \end{equation*}
    両辺を積分すると、\\
    \begin{eqnarray*}
      \int \frac{1}{1 + y ^ 2}dy & = & \int dx \\
      \arctan y & = & x + C \qquad (Cは積分定数)
    \end{eqnarray*}

    \newpage

    \qpart $ \displaystyle (x^2 + 1)\frac{dy}{dx} + y ^ 2 + 1 = 0, \qquad y(0) = 1 $ \\
    {\bf I} と同様に変数分離を行うと、\\
    \begin{equation*}
      \frac{1}{y ^ 2 + 1} \frac{dy}{dx} = - \frac{1}{x ^ 2 + 1}dx
    \end{equation*}
    両辺を積分すると、\\
    \begin{eqnarray*}
      & \arctan y & = - \arctan x + C \qquad (Cは積分定数) \\
      & \arctan y & + \arctan x = C
    \end{eqnarray*}
    となるので、両辺に $ \tan $ をとると、\\
    \begin{equation}
      \tan (\arctan y + \arctan x) = \tan C
    \end{equation}
    \begin{itembox}{$ \tan $ の加法定理}
      \begin{equation*}
        \tan(a + b) = \frac{\tan a + \tan b}{1 - \tan a \tan b}
      \end{equation*}
    \end{itembox}
    であるから、(0.0.1)式に当てはめると、\\
    \begin{equation}
      \tan (\arctan y + \arctan x) = \frac{y + x}{1 - xy}
    \end{equation}
    \begin{itembox}[l]{三角関数の逆関数との関係}
      \begin{eqnarray*}
        \sin(\arcsin x) & = & x \\
        \cos(\arccos x) & = & x \\
        \tan(\arctan x) & = & x \\
      \end{eqnarray*}
    \end{itembox}
    したがって、(0.0.1),(0.0.2)式より、\\
    \begin{equation*}
      \frac{x + y}{1 - xy} = \tan C
    \end{equation*}
    とが得られる。\\
    あとは、初期値条件に $ y(0) = 1 $ を代入すると、
    \begin{equation*}
      \tan C = 1
    \end{equation*}
    が得られるので、求める特殊解は、\\
    \begin{equation*}
      y = \frac{1 - x}{1 + x}
    \end{equation*}
  \end{qparts}
\end{spacing}

\end{document}
