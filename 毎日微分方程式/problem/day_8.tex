\documentclass[dvipdfmx,uplatex]{jsarticle}
\def\vector#1{\mbox{\boldmath $#1$}}
\usepackage{qexam} % 問題を書く時とかに必要なやつ
\usepackage{setspace} % 行間開けるのに必要なやつ
\usepackage{amsmath} % 数学やるのに必要なやつ
\usepackage{bm} % 太字にするのに必要なやつ
\usepackage{amssymb}
\usepackage[hiresbb]{graphicx}
\usepackage{ascmac}
\usepackage{siunitx}
\usepackage{float}
\usepackage{tikz}
\usepackage{circuitikz}
\usepackage{url}
\usepackage{braket}
\usepackage[colorlinks=true, bookmarks=true,
bookmarksnumbered = true, bookmarkstype = toc, linkcolor = blue,
urlcolor=blue, citecolor=blue]{hyperref}
\usepackage[version=3]{mhchem}
\makeatletter
 \renewcommand{\theequation}{
   \thesubsection.\arabic{equation}}
  \@addtoreset{equation}{subsection}
\title{春休み毎日微分方程式 Day 8(問題)}
\author{公立はこだて未来大学 システム情報科学部 B2 日置竜輔}
\date{\today}

\begin{document}
\begin{spacing}{1.6}
\maketitle

\question{問1}
{\bf ニュートンの冷却の法則}に基づいて、以下の問いに答えよ。\\
(必要ならば関数電卓を用いてもよい。)
\begin{qparts}
  \qpart
  ニュートンの冷却の法則によると、\\
  温度$T_0$の物質で囲まれた物質の温度$T(t)$の変化率は温度差$T(t) - T_0$に比例し、
  \begin{equation}
    \frac{dT}{dt} = -k(T - T_0) \qquad (k > 0)\nonumber
  \end{equation}
  が成り立つ。\\
  この時、$100$度の銅球を$20$度の液体に入れた。\\
  ただし、液体の質量の温度は銅球の質量に比べて、十分に大きいものとする。\\
  $3$分後に銅球の温度は$80$度になった。この時、銅球が$21$度になるのは何分後か。\\

  \end{qparts}
\end{spacing}

\end{document}
