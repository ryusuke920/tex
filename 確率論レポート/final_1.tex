\documentclass[12pt,a4paper]{jsarticle}
\usepackage {amsmath}
\usepackage {amssymb}
\usepackage {ascmac}
\title{確率論 宿題4の解答}
\date{}

\begin{document}
\maketitle
\begin{flushright}
    \author{1019163 2-G 日置竜輔}
\end{flushright}

次の問題を解き,解答過程を示したレポートを提出しなさい.\\
ただし,解答は単に答えを書くのではなく,どのように考えたかが分かるように書きなさい.

\begin{description}
    \item [問題] あるサイコロを n 回投げたとき,i の目が出る回数を Xi で表す.\\
    このとき,以下の問いに答えよ.\\
    但し、解答は単に答えを書くのではなく,どのように考えたかが分かるように書きなさい.
\end{description}

\begin{description}
    \item [問1] サイコロの偶数の目が奇数の目より2倍でやすいサイコロのXi とXj の同時確率P(Xi, Xj)
    が従う確率分布を求めよ.\\
    ただし,奇数となるどの目が出る確率も等しく,偶数となるどの目が出る確率も等しいとする.
\end{description}

まず、「偶数の目が奇数の目より2倍でやすい」ので、サイコロを1回振ったら\\

\left\{
    {\displaystyle} \begin{array}{ll}
        1,3,5の出る確率はそれぞれ & \frac{1}{9} \\
        2,4,6の出る確率はそれぞれ & \frac{2}{9} ~~~~であることがわかる。\\
    \end{array}
\right

~\\
(a)i~,~jが共に奇数であるとき、\\
さいころをn回振って、k回だけiが出る確率(すなわちX_{i} = k)は,\\

\begin{align*}
    {}_n C_k \left(\frac{1}{9}\right)^k\left(\frac{8}{9}\right)^{n-k}~~~~(反復試行の回数)\\
\end{align*}

また、サイコロをn回振って、k回だけiが出て、\\
l回だけjが出る確率(すなわちX_{i}=k~,~X_{j}=lの同時確率)は、\\

\begin{align*}
    \frac{n!}{k!~l!~(n-k-l)!}\left(\frac{1}{9}\right)^k \left(\frac{1}{9}\right)^l \left(\frac{7}{9}\right)^{n-k-l}\\
\end{align*}

したがって、i~,~jが共に奇数のときの表は以下のようになる。

\begin{table}[htb]
    \centering
    \caption{i,jが共に奇数のときの確率分布}
        \begin{array}{|c|c|c|c|c|} \hline
            (横:X_{i})/(縦:X_{j}) & 0 & 1 & 2 & \dots \\ \hline
            0 & \left(\frac{7}{9}\right)^n & n\frac{7^{n-1}}{9^n} & \frac{n(n-1)}{2}\frac{7^{n-2}}{9^n} & \dots \\ \hline
            1 & n\frac{7^{n-1}}{9^n} & n(n-1)\frac{7^{n-2}}{9^n} & \frac{n(n-1)(n-2)}{2}\frac{7^{n-3}}{9^n} & \dots \\ \hline
            2 & \frac{n(n-1)}{2}\frac{7^{n-2}}{9^n} & \frac{n(n-1)(n-2)}{2}\frac{7^{n-3}}{9^n} & \frac{n(n-1)(n-2)(n-3)}{4}\frac{7^{n-5}}{9^n} & \dots \\ \hline
            \dots & \dots & \dots & \dots & \dots \\ \hline
            l & \frac{n!}{l!(n-l)!}\frac{7^{n-l}}{9^n} & \frac{n!}{l!~(n-l-1)!}\frac{7^{n-k-1}}{9^n} & \frac{n!}{2!~l!~(n-l-2)!}\frac{7^{n-l-2}}{9^n} & \dots \\ \hline
            \dots & \dots & \dots & \dots & \dots \\ \hline
    \end{array}
\end{table}

\begin{table}[htb]
    \centering
        \begin{array}{|c|c|c|c|} \hline
            (横:X_{i})/(縦:X_{j}) & \dots & k & \dots \\ \hline
            0 & \dots & \frac{n!}{k!(n-k)!}\frac{7^{n-k}}{9^n} & \dots \\ \hline
            1 & \dots & \frac{n!}{k!~(n-k-1)!}\frac{7^{n-k-1}}{9^n} & \dots \\ \hline
            2 & \dots & \frac{n!}{2!~k!~(n-k-2)!}\frac{7^{n-k-2}}{9^n} & \dots \\ \hline
            \dots & \dots & \dots & \dots \\ \hline
            l & \dots & \frac{n!}{k!~l!~(n-k-l)!}\frac{7^{n-k-l}}{9^n} & \dots \\ \hline
            \dots & \dots & \dots & \dots \\ \hline
    \end{array}
\end{table}

~\\
(b)~iが奇数,~jが偶数であるとき、\\
さいころをn回振って、k回だけiが出る確率(すなわちX_{i} = k)は,\\

\begin{align*}
    {}_n C_k \left(\frac{1}{9}\right)^k\left(\frac{8}{9}\right)^{n-k}~~~~(反復試行の回数)\\
\end{align*}

また、サイコロをn回振って、k回だけiが出て、\\
l回だけjが出る確率(すなわちX_{i}=k~,~X_{j}=lの同時確率)は、\\

\begin{align*}
    \frac{n!}{k!~l!~(n-k-l)!}\left(\frac{1}{9}\right)^k \left(\frac{2}{9}\right)^l \left(\frac{2}{3}\right)^{n-k-l}\\
\end{align*}

したがって、~iが奇数,~jが偶数のときの表は以下のようになる。

\begin{table}[htb]
    \caption{~iが奇数,~jが偶数のときの確率分布}
        \begin{array}{|c|c|c|c|} \hline
            (横:X_{i})/(縦:X_{j}) & 0 & 1 & \dots \\ \hline
            
            0 & \left(\frac{2}{3}\right)^n & n\frac{1}{9}\left(\frac{2}{3}\right)^{n-1} &  \dots \\ \hline

            1 & n\frac{2}{9}\left(\frac{2}{3}\right)^{n-1} & n(n-1)\frac{2}{81}\left(\frac{2}{3}\right)^{n-2} & \dots \\ \hline
            
            2 & \frac{n(n-1)(n-2)}{2}\left(\frac{2}{9}\right)^2 \left(\frac{2}{3}\right)^{n-1} & \frac{n(n-1)(n-2)}{2}\frac{4}{9^3}\left(\frac{2}{3}\right)^{n-3} & \dots \\ \hline
            
            \dots & \dots & \dots & \dots \\ \hline
            
            l & \frac{n!}{l!~(n-l)!}\left(\frac{2}{9}\right)^l\left(\frac{2}{3}\right)^{n-2} & \frac{n!}{l!~(n-l-1)!}\frac{2^l}{9^{l+1}}\left(\frac{2}{3}\right)^{n-l-1} & \dots \\ \hline
            
            \dots & \dots  & \dots & \dots \\ \hline
    \end{array}
\end{table}

\begin{table}[htb]
    \centering
        \begin{array}{|c|c|c|c|c|} \hline
            (横:X_{i})/(縦:X_{j}) & \dots & 2 & k & \dots \\ \hline

            0 & \dots & \frac{n(n-1)}{2}\left(\frac{1}{9}\right)^2 \left(\frac{2}{3}\right)^{n-2} & \frac{n!}{k!~(n-k)!}\left(\frac{1}{9}\right)^k\left(\frac{2}{3}\right)^{n-k} & \dots \\ \hline
            
            1 & \dots & \frac{n(n-1)(n-2)}{2}\frac{2}{9^3}\left(\frac{2}{3}\right)^{n-3} & \frac{n!}{k!~(n-k-1)!}\frac{2}{9^{k+1}}\left(\frac{2}{3}\right)^{n-k-1} & \dots \\ \hline
            
            2 & \dots & \frac{n(n-1)(n-2)(n-3)}{4}\frac{4}{9^4}\left(\frac{2}{3}\right)^{n-4} & \frac{n!}{2!~k!~(n-k-2)!}\frac{4}{9^{k+2}}\left(\frac{2}{3}\right)^{n-k-2} & \dots \\ \hline
            
            \dots & \dots & \dots & \dots & \dots \\ \hline
            
            l & \dots & \frac{n!}{l!~(n-l-2)!}\frac{2^l}{9^{l+2}}\left(\frac{2}{3}\right)^{n-l-2} & \frac{n!}{k!~l!~(n-k-l)!}\frac{2^l}{9^{k+l}}\left(\frac{2}{3}\right)^{n-k-l} & \dots \\ \hline
            
            \dots & \dots & \dots & \dots & \dots \\ \hline
    \end{array}
\end{table}


~\\
(c)~iが偶数,~jが奇数であるとき、\\
さいころをn回振って、k回だけiが出る確率(すなわちX_{i} = k)は,\\

\begin{align*}
    {}_n C_k \left(\frac{2}{9}\right)^k\left(\frac{7}{9}\right)^{n-k}~~~~(反復試行の回数)\\
\end{align*}

また、サイコロをn回振って、k回だけiが出て、\\
l回だけjが出る確率(すなわちX_{i}=k~,~X_{j}=lの同時確率)は、\\

\begin{align*}
    \frac{n!}{k!~l!~(n-k-l)!}\left(\frac{2}{9}\right)^k \left(\frac{1}{9}\right)^l \left(\frac{2}{3}\right)^{n-k-l}\\
\end{align*}

したがって、~iが偶数,~jが奇数のときの表は以下のようになる。

\begin{table}[htb]
    \caption{~iが偶数,~jが奇数のときの確率分布}
        \begin{array}{|c|c|c|c|} \hline
            (横:X_{i})/(縦:X_{j}) & 0 & 1 & \dots \\ \hline

            0 & \left(\frac{2}{3}\right)^n & \frac{n!}{(n-1)!}\frac{2}{9}\left(\frac{2}{3}\right)^{n-1} & \dots \\ \hline

            1 & n\frac{1}{9}\left(\frac{2}{3}\right)^{n-1} & n(n-1)\frac{2}{81}\left(\frac{2}{3}\right)^{n-2} & \dots \\ \hline
            
            2 & \frac{n(n-1)}{2} \left(\frac{1}{9}\right)^2\left(\frac{2}{3}\right)^{n-2} & \frac{n(n-1)(n-2)}{2}\frac{2}{9^3}\left(\frac{2}{3}\right)^{n-3} & \dots \\ \hline
            
            \dots & \dots & \dots & \dots \\ \hline
            
            l & \frac{n!}{(n-l)!} \left(\frac{1}{9}\right)^l\left(\frac{2}{3}\right)^{n-l} & \frac{n!}{l!~(n-l-1)!}\frac{2}{9^{l+1}}\left(\frac{2}{3}\right)^{n-l-1} & \dots \\ \hline
            
    \end{array}
\end{table}


\begin{table}[htb]
    \centering
        \begin{array}{|c|c|c|c|c|} \hline
            (横:X_{i})/(縦:X_{j}) & \dots & 2 & k & \dots \\ \hline
            
            0 & \dots & \frac{n(n-1)}{2}\frac{4}{81}\left(\frac{2}{3}\right)^{n-2} & \frac{n!}{k!~(n-k)!}\left(\frac{2}{3}\right)^{n-k} & \dots \\ \hline

            1 & \dots & \frac{n(n-1)(n-2)}{2}\frac{4}{9^3}\left(\frac{2}{3}\right)^{n-3} & \frac{n!}{k!~(n-k-1)!}\frac{2^k}{9^{k+1}}\left(\frac{2}{3}\right)^{n-k-1} & \dots \\ \hline
            
            2 & \dots & \frac{n(n-1)(n-2)(n-3)}{4}\frac{4}{9^4}\left(\frac{2}{3}\right)^{n-4} & \frac{n!}{2!~k!~(n-k-2)!}\frac{2^k}{9^{k+2}}\left(\frac{2}{3}\right)^{n-k-2} & \dots \\ \hline
            
            \dots & \dots & \dots & \dots & \dots \\ \hline
            
            l & \dots & \frac{n!}{2!~l!~(n-l-2)!}\frac{4}{9^{l+2}}\left(\frac{2}{3}\right)^{n-l-2}  & \frac{n!}{k!~l!~(n-k-l)!}\frac{2^k}{9^{k+l}}\left(\frac{2}{3}\right)^{n-k-l} & \dots \\ \hline
            
            \dots & \dots & \dots & \dots & \dots \\ \hline
    \end{array}
\end{table}


~\\
(d)~i~,jが共に偶数であるとき、\\
さいころをn回振って、k回だけiが出る確率(すなわちX_{i} = k)は,\\

\begin{align*}
    {}_n C_k \left(\frac{2}{9}\right)^k\left(\frac{7}{9}\right)^{n-k}~~~~(反復試行の回数)\\
\end{align*}

また、サイコロをn回振って、k回だけiが出て、\\
l回だけjが出る確率(すなわちX_{i}=k~,~X_{j}=lの同時確率)は、\\

\begin{align*}
    \frac{n!}{k!~l!~(n-k-l)!}\left(\frac{2}{9}\right)^k \left(\frac{2}{9}\right)^l \left(\frac{5}{9}\right)^{n-k-l}\\
\end{align*}

したがって、~i,~jが共に偶数のときの表は以下のようになる。

\begin{table}[htb]
    \caption{~i,~jが共に偶数のときの確率分布}
        \begin{array}{|c|c|c|c|} \hline
            (横:X_{i})(縦:X_{j}) & 1 & 2 & \dots \\ \hline
            
            1 & n(n-1)4\frac{{5^{n-2}}}{9^n} & \frac{n(n-1)(n-2)}{2}8\frac{{5^{n-3}}}{9^n} & \dots \\ \hline
            
            2 &  \frac{n(n-1)(n-2)}{2}8\frac{{5^{n-3}}}{9^n} & \frac{n(n-1)(n-2)(n-3)}{4}16\frac{{5^{n-4}}}{9^n} & \dots \\ \hline
            
            \dots & \dots & \dots & \dots \\ \hline
            
            l &\frac{n!}{l!~(n-l-1)!}2^{l+1}\frac{{5^{n-l-1}}}{9^n} & \frac{n!}{2!~l!~(n-l-2)!}2^{l+2}\frac{5^{n-l-2}}{9^n} & \dots \\ \hline
            
            \dots & \dots & \dots & \dots \\ \hline
    \end{array}
\end{table}

\begin{table}[htb]
    \centering
        \begin{array}{|c|c|c|c|} \hline
            (横:X_{i})(縦:X_{j}) & \dots & k & \dots \\ \hline
            
            1 & \dots &  \frac{n!}{k!~(n-k-1)!}2^{k+1}\frac{{5^{n-k-1}}}{9^n} & \dots \\ \hline
            
            2 & \dots & \frac{n!}{2!~k!~(n-k-2)!}2^{k+2}\frac{5^{n-k-2}}{9^n} & \dots \\ \hline
            
            \dots & \dots & \dots & \dots \\ \hline
            
            l & \dots & \frac{n!}{k!~l!~(n-k-l)!}2^{k+l}\frac{5^{n-k-l}}{9^n} & \dots \\ \hline
            
            \dots & \dots & \dots & \dots \\ \hline
    \end{array}
\end{table}
\end{document}