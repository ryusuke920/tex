\documentclass[12pt,a4paper]{jsarticle}
\usepackage {amsmath}
\usepackage {amssymb}
\usepackage {ascmac}

\begin{document}

\begin{description}
    \item [問3] n が十分大きな値のとき,問 2 の周辺確率分布が収束する確率分布を求めよ.
\end{description}

{\displaystyle}nが十分大きな値をとる時、問2の周辺確率分布は正規分布に近似するため、正規分布の公式に当てはめる。\\
\mu  = \frac{n}{9}~,~\sigma^2 = \mu(1-\mu) =\frac{8}{81}nであるから、正規分布は\\

\begin{align*}
    {\displaystyle}f(l)~&=~\frac{1}{\sqrt{2{\pi}{\sigma}^2}}\exp\left\{-\frac{(x ~-~ \mu)^2}{2\sigma^2}\right\} \\
    &= \frac{1}{\sqrt{2{\pi}\frac{8}{81}n}}\exp\left\{-\frac{(x-\frac{n}{9})^2}{2\frac{8}{81}n}\right\} \\
    &= \frac{9}{4\sqrt{n{\pi}}} \exp \left\{\frac{81}{16n}\left(x-\frac{n}{9}\right)^2\right\}
\end{align*}

したがって、求める確率分布は

\begin{align*}
    f(l) ~=~ \frac{9}{4\sqrt{n{\pi}}} \exp \left\{\frac{81}{16n}\left(x-\frac{n}{9}\right)^2\right\}
\end{align*}

\end{document}