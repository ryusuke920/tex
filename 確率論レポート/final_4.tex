\documentclass[12pt,a4paper]{jsarticle}
\usepackage {amsmath}
\usepackage {amssymb}
\usepackage {ascmac}

\begin{document}

\begin{description}
    \item [問4] n = 25920000, X3 = 2877000であったとき,このサイコロが,偶数の目が奇数の目より2倍
    以上でやすいサイコロであるかどうかを,有意水準 5{\%}で検定せよ.\\
    ただし,n = 25920000は十分大きな値であると見做し,問3の確率分布に収束していると見做してよい.\\
    さらに,奇数となるどの目が出る確率も等しく,偶数となるどの目が出る確率も等しいとする.
\end{description}

このサイコロが、普通のサイコロ(1~〜~6が等確率で出る)と仮定すると、\\
X_{3}の確率分布は、nが十分大きいならば、\\

\begin{align*}
    &P_{(x,n)} ~=~ \frac{6}{\sqrt{10n{\pi}}} \exp\left\{ -\frac{8}{15n}\left(x - \frac{n}{6}\right)^2  \right\}に収束する。\\
    &したがって、n ~=~ 2,592,0000~より、\frac{n}{6} ~=~ 4,320,000 \\
    &P_{(x)} ~=~ \frac{1}{1200\sqrt{5\pi}}\exp\left\{ -\frac{1}{7.2~×~10^6}\left(x-4.32×10^6\right)^2 \right\} となる。\\
    &また、\sigma ~=~ 600\sqrt{10} ~=~ 1897.3\dots ~~,~~ 2\sigma ~=~ 3794.7\dots \\
    &\frac{n}{6}-20 ~=~ 4,316,205.2\dots ~~,~~ \frac{n}{6}+20 ~=~ 4,323,794.7\dots である。\\
\end{align*}

以上より、~~4,316,205~\leqq~X_{3}~\leqq~4,323,795となる確率はが~95.4~\%~以上なので、\\
このサイコロは偶数の目が奇数の目より2倍以上でやすいサイコロである可能性が高い。\\

\end{document}