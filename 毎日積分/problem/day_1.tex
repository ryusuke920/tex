\documentclass[dvipdfmx,uplatex]{jsarticle}
\def\vector#1{\mbox{\boldmath $#1$}}
\usepackage{qexam} % 問題を書く時とかに必要なやつ
\usepackage{setspace} % 行間開けるのに必要なやつ
\usepackage{amsmath} % 数学やるのに必要なやつ
\usepackage{bm} % 太字にするのに必要なやつ
\usepackage{cases} % 連立方程式を書くのに必要なやつ
\usepackage{amssymb}
\usepackage[hiresbb]{graphicx}
\usepackage{ascmac}
\usepackage{siunitx}
\usepackage{float}
\usepackage{tikz}
\usepackage{circuitikz}
\usepackage{url}
\usepackage{braket}
\usepackage[colorlinks=true, bookmarks=true,
bookmarksnumbered = true, bookmarkstype = toc, linkcolor = blue,
urlcolor=blue, citecolor=blue]{hyperref}
\usepackage[version=3]{mhchem}
\makeatletter
 \renewcommand{\theequation}{
   \thesubsection.\arabic{equation}}
  \@addtoreset{equation}{subsection}
\title{夏休み毎日積分 Day1(問題)}
\author{公立はこだて未来大学 システム情報科学部 B2 日置竜輔}
\date{2020年8月1日}

\begin{document}

\maketitle

\begin{itembox}[c]{day 1 }
    \begin{center}
        次の不定積分を求めよ。\quad (積分定数はCとする)\\
        ※とりあえず初日は自分も作り方を慣れていないので、公式の確認としてめちゃ簡単にしました。
    \end{center}
\end{itembox}
\begin{align*}
    &\int sinx dx \\
    &\int cosx dx \\
    &\int \frac{1}{x} dx \\
    &\int x^a dx \\
    &\int logx dx \\
    &\int e^x dx \\
    &\int \frac{1}{cos^2x} dx \\
    &\int \frac{1}{sin^2x} dx \\
    &\int a^x dx
\end{align*}
\end{document}
