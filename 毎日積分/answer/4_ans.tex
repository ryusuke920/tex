\documentclass[12pt,a4paper]{jsarticle}
\usepackage {amsmath}
\usepackage {ascmac}
\title{夏休み毎日積分~4日目(解答)}
\date{\today}

\begin{document}
\maketitle

\begin{flushright}
    \author{作成者:ryusuke.h}
\end{flushright}

\begin{itembox}[c]{day 4 }
    \begin{center}
        次の積分を求めよ。\\
        ※今日から少しずつレベル上げていきます〜
    \end{center}
\end{itembox}

\begin{description}
    \item [問1] {\displaystyle}\int \frac{1}{\tan^2{2x}} dx \\
\end{description}

\begin{equation*}
    \tan^2{2x}+1 = \frac{1}{\cos^2{2x}}であるから、両辺の逆数を取ると、\\
    \frac{1}{\tan^2{2x}} = \cos^2{2x} - 1となる。
\end{equation*}

\begin{align*}
    さらに、\cos^2{2x} = \cos{2x}\cos{2x}&であるから、積和の公式を使用すると、\\
    2\cos{2x}\cos{2x} &= \cos4x + \cos0 \\
    \cos^2{2x} &= \frac{1}{2}(cos4x+cos0) \\
    \cos^2{2x} &= \frac{1}{2}(1 + cos4x)
\end{align*}

したがって、\\
\begin{align*}
    \frac{1}{\tan^2{2x}} = \frac{1}{2}(1+cos4x) - 1 = \frac{1}{2}\cos4x - \frac{1}{2}であるから、
\end{align*}
\begin{align*}    
    (与式) = \int\frac{1}{2}\cos4x - \frac{1}{2} dx
    &= \frac{1}{2}\int\cos4xdx - \frac{1}{2}{\int}dx \\
    &= \frac{1}{8}\sin4x - \frac{1}{2}x + C ~~~(Cは積分定数とする)\\
\end{align*}

\begin{description}
    \item [問2] \int \sin^2{2x} dx
\end{description}

\begin{equation*}
    \sin^2{2x} = \sin{2x}\sin{2x}であるから、積和の公式を再び使用すると、
\end{equation*}

\begin{align*}
    -2\sin{2x}\sin{2x} &= \cos(2x+2x) – \cos(2x-2x) \\
    -2\sin^2{2x} &= \cos{4x} - \cos{0} \\
    \sin^2{2x} &= -\frac{1}{2}(\cos4x - 1) \\
    \sin^2{2x} &= \frac{1}{2}(1 - \cos{4x})となる。 \\
\end{align*}

したがって、\\
\begin{align*}    
    (与式) = \int\frac{1}{2}(1 - \cos{4x}) dx
    &= \frac{1}{2}(\int{dx} - \int\cos{4x})dx \\
    &= \frac{1}{2}(x - \frac{1}{4}\sin{4x}) + C \\
    &= \frac{1}{2}x - \frac{1}{8}\sin{4x} + C ~~~(Cは積分定数とする)
\end{align*}

\begin{boxnote}
    〜補足〜
    \begin{center}
        積和のの問題再来です。\\
        気づければ瞬殺だけどわからないと手が出せないものですよね、 \\
        解説でわからなければ自分で調べるか、個別に聞いてください。
    \end{center}
\end{boxnote}

\end{document}