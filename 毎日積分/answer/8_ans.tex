\documentclass[12pt,a4paper]{jsarticle}
\usepackage {amsmath}
\usepackage {amssymb}
\usepackage {ascmac}
\title{夏休み毎日積分~8日目(解答)}
\date{\today}

\begin{document}
\maketitle

\begin{flushright}
    \author{作成者:ryusuke.h}
\end{flushright}

\begin{itembox}{day 8 }
    \begin{center}
        次の定積分を求めよ。\\
        ※今日はガウス積分です。
    \end{center}
\end{itembox}

\begin{description}
    \item [問1] {\displaystyle} \iint_R e^{\frac{y}{x}} dxdy ~~~~ (R = \{(x,y); 0 \leq x \leqq 1, 0 \leqq y \leqq x\}); \\
\end{description}

求める積分値をIとすると、

\begin{align*}
    I &= \int_0^1 \left( \int_0^x e^\frac{y}{x} dy\right) dx \\
    &= \int_0^1 \left[ xe^\frac{y}{x}\right]_0^x = \int_0^1 (xe - x) dx \\
    &= \left[ \frac{x^2}{2}(e - 1)\right]_0^1 = \frac{1}{2}(e - 1)
\end{align*}

\begin{description}
    \item [問2] {\displaystyle} \iint_D \sqrt{a^2 - x^2 - y^2} dxdy ~~~~ (D = \{(x,y); x^2 + y^2 \leqq a^2\}); \\
\end{description}

求める積分値をIとして、極座標に変換すると、

\begin{align*}
    I &= \int_0^a\int_0^{2\pi} \sqrt{a^2 - r^2} rdrd\theta \\
    &= \int_0^a \sqrt{a^2 - r^2} r \left(\int_0^{2\pi} d\theta \right) = 2\pi \int_0^a \sqrt{a^2 - r^2}rdr \\
    &= \frac{2\pi}{3}\left[ - \left(a^2 - r^2\right)^\frac{3}{2}\right]_0^a = \frac{2}{3}{\pi}a^2
\end{align*}

\begin{boxnote}
    〜補足〜
    \begin{center}
        重積分を行いました。\\
        出典は名古屋大学の解析学の期末試験の簡単な部分の問題です。\\
        解説でわからなければ自分で調べるか、個別に聞いてください。
    \end{center}
\end{boxnote}

\end{document}