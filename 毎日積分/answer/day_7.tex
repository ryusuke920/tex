\documentclass[12pt,a4paper]{jsarticle}
\usepackage {amsmath}
\usepackage {amssymb}
\usepackage {ascmac}
\title{夏休み毎日積分~7日目(解答)}
\date{\today}

\begin{document}
\maketitle

\begin{flushright}
    \author{作成者:ryusuke.h}
\end{flushright}

\begin{itembox}{day 7 }
    \begin{center}
        次の定積分を求めよ。\\
        ※今日はガウス積分です。
    \end{center}
\end{itembox}

\begin{description}
    \item [問1] {\displaystyle} \int_{-\infty}^{\infty} e^{-ax^2} dx \\
\end{description}

求める積分値をIとおく。

\begin{align*}
    I^2 &= \left(\int_{-\infty}^{\infty} e^{-x^2} dx\right)^2 
    =  \int_{-\infty}^{\infty} e^{-x^2} dx \int_{-\infty}^{\infty} e^{-y^2} dy\\
\end{align*}

ここで x = r\cos\theta , y = r\sin\theta と置換すると,ヤコビアンは r なので,

\begin{align*}
    I^2 &= \int_0^{\infty} \int_0^2{\pi} e^{-ar^2} r d{\theta}dr 
    = 2{\pi} \int_0^{\infty} e^{-ar^2} r dr \\
    &= 2{\pi} \left[\frac{e^{-ar^2}}{-2a}\right]_0^{\infty} \\
    &= \frac{\pi}{a}
\end{align*}

\begin{description}
    \item [問2] {\displaystyle} \int_{-\infty}^{\infty} e^{-a(x-b)^2} dx
\end{description}

平行移動しても積分値は変わらない(x軸方向に+b平行移動)ので、グラフの形は不変であるので、

\begin{align*}
    \int_{-\infty}^{\infty} e^{-a(x-b)^2} dx = \frac{\pi}{a}
\end{align*}

\begin{description}
    \item [問3] {\displaystyle} \int_0^{\infty} e^{-ax^2} dx
\end{description}

積分区間が問1の半分であるから、

\begin{align*}
    \int_0^{\infty} e^{-a(x-b)^2} dx = \frac{\pi}{2a}
\end{align*}

\begin{boxnote}
    〜補足〜
    \begin{center}
        ガウス積分を行いました。\\
        確率論では頻出なので覚えましょう。\\
        解説でわからなければ自分で調べるか、個別に聞いてください。
    \end{center}
\end{boxnote}

\end{document}