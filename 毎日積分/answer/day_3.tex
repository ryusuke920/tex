\documentclass[12pt,a4paper]{jsarticle}
\usepackage {amsmath}
\usepackage {ascmac}
\title{夏休み毎日積分~3日目(解答)}
\date{\today}

\begin{document}
\maketitle
\begin{flushright}
    \author{作成者:ryusuke.h}
\end{flushright}
\begin{itembox}[c]{day 3 }
    \begin{center}
        次の積分を求めよ。\\
        ※今日から少しずつレベル上げていきます〜
    \end{center}
\end{itembox}
\begin{description}
    \item [問1] {\displaystyle}\int \frac{3x^3+12x+1}{x^2+4} dx \\
\end{description}

とりあえず分子の次数を下げてみましょう。\\
\begin{align*}
\frac{3x^3+12x+1}{x^2+4} 
&=\frac{3x(x^2+4)}{x^2+4}+\frac{1}{x^2+4}~=~3x+\frac{1}{x^2+4}となるので、
\end{align*}

\begin{align*}
    \int \frac{3x^3+12x+1}{x^2+4} dx
    &= 3x+\frac{1}{x^2+4} dx \\
    &= \int{3x}dx + \int\frac{1}{x^2+4} dx
\end{align*}

\begin{equation*}
    \int\frac{1}{x^2+4}dxについて、x = 2\tan\frac{\theta}{2}と置くと、
\end{equation*}

\begin{equation*}
    \frac{x}{2}=\tan\frac{\theta}{2},~~dx = \frac{1}{\cos^2\frac{\theta}{2}}であるから、
\end{equation*}
\begin{align*}
    \int\frac{1}{x^2+4}dx
    &=\int\frac{1}{4}\frac{1}{\tan^2\frac{\theta}{2}+1}\frac{1}{\cos^2\frac{\theta}{2}} d\theta \\
    &= \int\frac{1}{4}d\theta\\
    &= \frac{1}{4}\theta
\end{align*}

\begin{equation*}    
\frac{\theta}{2} &= \arctan\frac{x}{2}より、\\
\frac{1}{4}\theta &= \frac{1}{2}\arctan\frac{x}{2}
\end{equation*}

したがって、\\
\begin{align*}    
    (与式) = \frac{3}{2}x^2+\frac{1}{2}\arctan\frac{x}{2}+C ~~~(Cは積分定数とする) \\\\
\end{align*}

\begin{description}
    \item [問2] {\displaystyle}\int_\frac{1}{2}^4 \frac{1}{8x+3} dx
\end{description}
\begin{align*}
    \intertext{\int \frac{1}{f(x)} dx = \log{f(x)}\frac{1}{f'(x)} + C ~~であることを考えると、}
    \int_\frac{1}{2}^4  \frac{1}{8x+3} dx  
    &= \frac{1}{8}[\log(8x+3)]_\frac{1}{2}^4 \\
    &= \frac{1}{8}\{\log(32+3)-\log(4+3)\} \\
    &= \frac{1}{8}(\log35 - \log7) \\
    &= \frac{1}{8}\log5
\end{align*}

\begin{boxnote}
    〜補足〜
    \begin{center}
        次数下げの問題です。\\
        分子の方が大きい時はこのような工夫をするようにしましょう。 \\
        解説でわからなければ自分で調べるか、個別に聞いてください。
    \end{center}
\end{boxnote}

\end{document}