\documentclass[12pt,a4paper]{jsarticle}
\usepackage {amsmath}
\usepackage {amssymb}
\usepackage {ascmac}
\title{夏休み毎日積分~6日目(解答)}
\date{\today}

\begin{document}
\maketitle

\begin{flushright}
    \author{作成者:ryusuke.h}
\end{flushright}

\begin{itembox}[c]{day 6 }
    \begin{center}
        次の定積分を求めよ。\\
        ※今日から本格的に難しくなります。\\
        解けたら十分自信を持っていいと思います。
    \end{center}
\end{itembox}

\begin{description}
    \item [問1] {\displaystyle}\int_0^2 \frac{2x + 1}{\sqrt{x^2 + 4}} dx
\end{description}

\begin{align*}
    \int_0^2 \frac{2x + 1}{\sqrt{x^2 + 4}} dx &= \int_0^2 \frac{2x}{\sqrt{x^2 + 4}} dx + \int_0^2 \frac{1}{\sqrt{x^2 + 4}} dx 
    \intertext{と分けて考える。}
\end{align*}

前半について、

\begin{equation*}
    t = x^2 + 4~ とおくと、~dt = 2xdx~ であり、x は~0 → 2~のとき、t~は4 → 8~であるから、\\
\end{equation*}

\begin{align*}
    \int_0^2 \frac{2x}{\sqrt{x^2 + 4}} dx &= \int_4^8 \frac{1}{\sqrt{t}} dt \\
    &= [2\sqrt{t}]_4^8 \\
    &= 4\sqrt{2} - 4~となる。
\end{align*}

後半について、

\begin{equation*}
    x = \tan\theta~と置くと、~d\theta = \frac{2}{\cos^2\theta} d\theta ~であり、~ x は~0 → 2~のとき、\theta~は0 → \frac{\pi}{4}~であるから、\\
\end{equation*}

\begin{align*}
    \int_0^2 \frac{1}{\sqrt{x^2 + 4}} dx &= \int_0^\frac{\pi}{4} \frac{1}{2\sqrt{{\tan}^2\theta + 1}} \frac{2}{\cos^2\theta} d\theta \\
    &= \int_0^\frac{\pi}{4} \frac{1}{\cos\theta} d\theta \\
    &= \int_0^\frac{\pi}{4} \frac{\cos\theta}{1 - \sin^2\theta} d\theta
\end{align*}

\begin{equation*}
    さらにここで、t = \sin\theta~と置くと、~dt = \cos\theta d\theta ~であり、~ \theta は~0 → \frac{\pi}{4}~のとき、t~は0 → \frac{1}{\sqrt{2}}~であるから、\\
\end{equation*}

\begin{align*}
    \int_0^2 \frac{1}{\sqrt{x^2 + 4}} dx &= \int_0^\frac{1}{\sqrt{2}} \frac{1}{1 - t^2} dt \\
    &= \int_0^\frac{1}{\sqrt{2}} \frac{1}{2}(\frac{1}{1 - t} + \frac{1}{1 + t}) dt \\
    &= \frac{1}{2} [\log{\left}| \frac{1 + t}{1 - t} {\right}|]_0^\frac{1}{\sqrt{2}} \\
    &= \frac{1}{2}\log\frac{\sqrt{2}+ 1}{\sqrt{2} - 1} \\
    &= \frac{1}{2}\log(\sqrt{2} + 1)^2 \\
    &= \log(\sqrt{2} + 1)
\end{align*}

\begin{description}
    \item [問2] {\displaystyle} \int_0^2 \left|x^2 - a^2\right| dx をaを用いて表せ。\\
\end{description}

\intertext{{&}解法について、\\{&}{\left}x = a\right が積分区間 0 \leqq~ x \leqq~ 2 に含まれているか否かで場合分けをする。\\}
\intertext{(a) 0 \leq a\leqq 2 のとき、}

\begin{align*}
    (与式) &= \int_0^a \{ - (x^2 - a^2)\} dx + \int_a^2 (x^2 - a^2) dx \\
    &= \left[-\frac{x^3}{3} + a^2x\right]_0^a + \left[\frac{x^3}{3} - a^2\right]_a^2 \\
    &= \frac{2}{3}a^3 + (\frac{8}{3} - 2a^2 + \frac{2}{3}a^3) \\
    &= \frac{4}{3}a^3 - 2a^2 + \frac{8}{3}
\end{align*}

\intertext{(b)  a {\geqq} 2 のとき、}

\begin{align*}
    (与式) &= \int_0^2 \{- (x^2 - a^2)\} dx \\ 
    &= \left[ - \frac{x^3}{3} + a^2x\right]_0^2 \\
    &= 2a^2 - \frac{8}{3}
\end{align*}

したがって、(a), (b) より\\

\begin{cases}
    {\left}\frac{4}{3}a^3 - 2a^2 + \frac{8}{3} {\right}~~~~ (0 ~\leq a~ \leqq 2~のとき) \\
    {\left}2a^2 - \frac{8}{3} {\right}~~~~(a~\geqq~2~のとき) \\
\end{cases}

\begin{boxnote}
    〜補足〜
    \begin{center}
        問1は難問でしたが、問2は単純な絶対値の場合分けの問題です。\\
        中身の正負で場合分けをすれば終わりですね。\\
        解説でわからなければ自分で調べるか、個別に聞いてください。
    \end{center}
\end{boxnote}

\end{document}