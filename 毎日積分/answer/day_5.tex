\documentclass[12pt,a4paper]{jsarticle}
\usepackage {amsmath}
\usepackage {ascmac}
\title{夏休み毎日積分~5日目(解答)}
\date{\today}

\begin{document}
\maketitle

\begin{flushright}
    \author{作成者:ryusuke.h}
\end{flushright}

\begin{itembox}[c]{day 5 }
    \begin{center}
        次の定積分を求めよ。\\
        ※本日は難問なので1問のみの出題です。頑張ってください。
    \end{center}
\end{itembox}

\begin{description}
    \item [問1] {\displaystyle}\int_0^1 \log(x^2 + 1) dx 
\end{description}

\begin{align*}
    \int_0^1 \log(x^2 + 1) dx &= \int_0^1 (x)'\log(x^2 + 1) dx \\
        &= [x\log(x^2 + 1)]_0^1 - \int_0^1 x\frac{2x}{x^2 + 1} dx \\
        &= \log2 - 2\int_0^1\frac{x^2}{x^2 + 1} dx \\
        &= \log2 - 2\int_0^1 (1 - \frac{1}{x^2+1} ) dx
\end{align*}

\begin{equation*}
    ここで、 ~x = \tan{\theta}~とおくと、 dx = \frac{1}{\cos^2\theta}d{\theta}~であり、x = 0 → 1 ~~,~~\theta = 0 → \frac{\pi}{4}より、
\end{equation*}

\begin{align*}
    \int_0^1 \frac{1}{x^2 + 1} dx &= \int_0^\frac{\pi}{4} \frac{1}{\tan^2\theta + 1}\frac{d\theta}{\cos^2\theta} \\
    &= \int_0^\frac{\pi}{4} d\theta = \frac{\pi}{4}
\end{align*}

したがって、

\begin{align*}
    \log2 - 2\int_0^1 (1 - \frac{1}{x^2 + 1}) dx &= \log2 - 2([x]_0^1 - \frac{\pi}{4}) \\
    &= \log2 - 2 + \frac{\pi}{2}
\end{align*}

\begin{boxnote}
    〜補足〜
    \begin{center}
        (x)'を補って、部分積分を行うという手法です\\
        単独のlogは部分積分を行うのが常套手段です。 \\
        解説でわからなければ自分で調べるか、個別に聞いてください。
    \end{center}
\end{boxnote}

\end{document}