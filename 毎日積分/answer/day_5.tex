\documentclass[dvipdfmx,uplatex]{jsarticle}
\def\vector#1{\mbox{\boldmath $#1$}}
\usepackage{qexam} % 問題を書く時とかに必要なやつ
\usepackage{setspace} % 行間開けるのに必要なやつ
\usepackage{amsmath} % 数学やるのに必要なやつ
\usepackage{bm} % 太字にするのに必要なやつ
\usepackage{cases} % 連立方程式を書くのに必要なやつ
\usepackage{amssymb}
\usepackage[hiresbb]{graphicx}
\usepackage{ascmac}
\usepackage{siunitx}
\usepackage{float}
\usepackage{tikz}
\usepackage{circuitikz}
\usepackage{url}
\usepackage{braket}
\usepackage[colorlinks=true, bookmarks=true,
bookmarksnumbered = true, bookmarkstype = toc, linkcolor = blue,
urlcolor=blue, citecolor=blue]{hyperref}
\usepackage[version=3]{mhchem}
\makeatletter
 \renewcommand{\theequation}{
   \thesubsection.\arabic{equation}}
  \@addtoreset{equation}{subsection}
\title{夏休み毎日積分 Day5(問題)}
\author{公立はこだて未来大学 システム情報科学部 B2 日置竜輔}
\date{2020年8月5日}
\begin{document}

\maketitle

\begin{itembox}[c]{day 5}
    \begin{center}
        次の定積分を求めよ。\\
        ※本日は難問なので1問のみの出題です。頑張ってください。
    \end{center}
\end{itembox}

\begin{description}
    \item [問1] $ \displaystyle \int_0^1 \log(x^2 + 1) dx $
\end{description}

\begin{align*}
    \int_0^1 \log(x^2 + 1) dx &= \int_0^1 (x)'\log(x^2 + 1) dx \\
        &= [x\log(x^2 + 1)]_0^1 - \int_0^1 x\frac{2x}{x^2 + 1} dx \\
        &= \log2 - 2\int_0^1\frac{x^2}{x^2 + 1} dx \\
        &= \log2 - 2\int_0^1 (1 - \frac{1}{x^2+1} ) dx
\end{align*}
ここで、 $ \displaystyle x = \tan{\theta} $とおくと、
\begin{equation*}
  dx = \frac{1}{\cos^2\theta}d{\theta}
\end{equation*}
であり、
\begin{equation*}
  x = 0 → 1 ~~,~~\theta = 0 → \frac{\pi}{4}
\end{equation*}
より、
\begin{align*}
    \int_0^1 \frac{1}{x^2 + 1} dx &= \int_0^\frac{\pi}{4} \frac{1}{\tan^2\theta + 1}\frac{d\theta}{\cos^2\theta} \\
    &= \int_0^\frac{\pi}{4} d\theta = \frac{\pi}{4}
\end{align*}

したがって、

\begin{align*}
    \log2 - 2\int_0^1 (1 - \frac{1}{x^2 + 1}) dx &= \log2 - 2([x]_0^1 - \frac{\pi}{4}) \\
    &= \log2 - 2 + \frac{\pi}{2}
\end{align*}

\begin{boxnote}
    〜補足〜
    \begin{center}
        (x)'を補って、部分積分を行うという手法です\\
        単独のlogは部分積分を行うのが常套手段です。 \\
        解説でわからなければ自分で調べるか、個別に聞いてください。
    \end{center}
\end{boxnote}

\end{document}
