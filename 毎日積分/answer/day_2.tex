\documentclass[dvipdfmx,uplatex]{jsarticle}
\def\vector#1{\mbox{\boldmath $#1$}}
\usepackage{qexam} % 問題を書く時とかに必要なやつ
\usepackage{setspace} % 行間開けるのに必要なやつ
\usepackage{amsmath} % 数学やるのに必要なやつ
\usepackage{bm} % 太字にするのに必要なやつ
\usepackage{cases} % 連立方程式を書くのに必要なやつ
\usepackage{amssymb}
\usepackage[hiresbb]{graphicx}
\usepackage{ascmac}
\usepackage{siunitx}
\usepackage{float}
\usepackage{tikz}
\usepackage{circuitikz}
\usepackage{url}
\usepackage{braket}
\usepackage[colorlinks=true, bookmarks=true,
bookmarksnumbered = true, bookmarkstype = toc, linkcolor = blue,
urlcolor=blue, citecolor=blue]{hyperref}
\usepackage[version=3]{mhchem}
\makeatletter
 \renewcommand{\theequation}{
   \thesubsection.\arabic{equation}}
  \@addtoreset{equation}{subsection}
\title{夏休み毎日積分 Day2(解答)}
\author{公立はこだて未来大学 システム情報科学部 B2 日置竜輔}
\date{2020年8月2日}

\begin{document}

\maketitle

\begin{itembox}[c]{day 2}
    \begin{center}
        次の積分を求めよ。\\
        ※少しずつレベル上げていきます〜
    \end{center}
\end{itembox}

\begin{qparts}
  \qpart
  \begin{equation*}
    \int \sin 3x \cos 5x dx
  \end{equation*}

和積の公式を用いて考えると、
\begin{eqnarray*}
    \int \sin 3x \cos 5xdx
    & = & \frac{1}{2} \int \{ \sin (3x+5x) + \sin (3x-5x) \} dx \\
    & = & \frac{1}{2} \int \sin 8x + \sin (-2x) dx \\
    & = & \frac{1}{2} \int (\sin 8x - \sin 2x) dx \\
    & = & \frac{1}{2} \left\{ \frac{1}{8} (-\cos 8x) - \frac{1}{2} (-\cos 2x) \right\} + C \\
    & = & -\frac{1}{16} \cos 8x + \frac{1}{4} \cos 2x + C \quad (Cは積分定数とする)
\end{eqnarray*}

\newpage

  \qpart
  \begin{equation*}
    \int \cos 3x \cos 5x dx
  \end{equation*}

問1と同様に和積の公式を用いて考えると、
\begin{eqnarray*}
    \int \cos 3x \cos 5x dx & = & \frac{1}{2} \int \left\{ \cos (3x+5x) + \cos (3x - 5x) \right\} dx \\
    & = & \frac{1}{2} \int \cos 8x + \cos (-2x) dx \\
    & = & \frac{1}{2} \int \left( \cos 8x + \cos 2x \right) dx \\
    & = & \frac{1}{2} \left( \frac{1}{8} \sin 8x - \frac{1}{2} \sin 2x \right) + C \\
    & = & \frac{1}{16}\sin 8x + \frac{1}{4} \sin 2x + C \quad (Cは積分定数とする)
\end{eqnarray*}
\end{qparts}

\begin{boxnote}
    〜補足〜
    \begin{center}
        今日は和積(積和ですね)の問題を出題しました。\\
        去年の解析レベルの問題なので、ぜひできるようにしてください。 \\
        わからなければ自分で調べるか、個別に聞いてください。
    \end{center}
\end{boxnote}
\end{document}
