\documentclass[12pt,a4paper,dvipdfmx]{jsarticle}

\usepackage {amsmath}
\usepackage {amssymb}
\usepackage {ascmac}
\usepackage {tikz}
\usetikzlibrary{intersections, calc, arrows}

\title{夏休み毎日積分~11日目(解答)}
\date{\today}

\begin{document}
\maketitle

\begin{flushright}
    \author{作成者:ryusuke.h}
\end{flushright}

\begin{itembox}{day 11 }
    \begin{center}
        次の定積分を求めよ。\\
        今日も今日とて重積分の問題です。\\
        飽きてきたと思うので、明日は変わった問題を出題してみます。
    \end{center}
\end{itembox}

\begin{description}
    \item [問1] {\displaystyle} \int_0^\infty \frac{1}{e^x + e^{-x}} ~dx~ \\
\end{description}

e^x ~=~ t~とおいて解いていく。

\begin{align*}
    \int \frac{1}{e^x+e^{-x}} dx &= \int \frac{e^x}{e^{2x}+1}dx ~=~ \frac{1}{1+t^2}dt 
    ~=~ \arctan{t} であるから、
\end{align*}

\begin{align*}
    (与式) ~=~  \left[\arctan{t}\right]_1^{\infty} ~=~ \frac{\pi}{2} ~-~ \frac{\pi}{4} ~=~ \frac{\pi}{4}
\end{align*}

\begin{description}
    \item [問2] {\displaystyle} \iint_D \frac{1}{1+x+y+xy} ~ dxdy ~~~~ D = \{(x,y) ~|~ 1 \leqq x \leqq 4,~ 2 \leqq y \leqq 3\}
\end{description}

\begin{align*}
    (与式) &= \int_1^4 dx \int_2^3 \frac{1}{(1+x)(1+y)}dy ~=~ \int_1^4dx \left[\frac{\log(1+y)}{1+x}\right]_{y = 2}^{y = 3} \\
    &= \int_1^4 \frac{\log{4} - \log{3}}{1+x}dx = \left[\left( \log{4}-\log{3} \right)\log\left(1+x\right)\right]_1^4 \\
    &= \left(\log{4}-\log{3}\right)\left(\log{5}-\log{2}\right)
\end{align*}

\begin{boxnote}
    〜補足〜
    \begin{center}
        1問目はe^xをかけることで、きれいな形になります。\\
        2問目は因数分解するということが見抜ければ、一発です。\\
        解説でわからなければ自分で調べるか、個別に聞いてください。
    \end{center}
\end{boxnote}

\end{document}