\documentclass[dvipdfmx,uplatex]{jsarticle}
\def\vector#1{\mbox{\boldmath $#1$}}
\usepackage{qexam} % 問題を書く時とかに必要なやつ
\usepackage{setspace} % 行間開けるのに必要なやつ
\usepackage{amsmath} % 数学やるのに必要なやつ
\usepackage{bm} % 太字にするのに必要なやつ
\usepackage{ascmac} % 影付きの四角で覆いたい時に必要なやつ
\usepackage{amssymb}
\usepackage[hiresbb]{graphicx}
\usepackage{ascmac}
\usepackage{siunitx}
\usepackage{float}
\usepackage{tikz}
\usepackage{circuitikz}
\usepackage{url}
\usepackage{braket}
\usepackage[colorlinks=true, bookmarks=true,
bookmarksnumbered = true, bookmarkstype = toc, linkcolor = blue,
urlcolor=blue, citecolor=blue]{hyperref}
\usepackage[version=3]{mhchem}
\makeatletter
 \renewcommand{\theequation}{
   \thesubsection.\arabic{equation}}
  \@addtoreset{equation}{subsection}
\title{春休み毎日微分方程式 Day 8(解答)}
\author{公立はこだて未来大学 システム情報科学部 B2 日置竜輔}
\date{2021年3月15日}

\begin{document}
\begin{spacing}{1.6}
\maketitle

\question{問1}
{\bf ニュートンの冷却の法則}に基づいて、以下の問いに答えよ。\\
(必要ならば関数電卓を用いてもよい。)
\begin{qparts}
  \qpart
  ニュートンの冷却の法則によると、\\
  温度$T_0$の物質で囲まれた物質の温度$T(t)$の変化率は温度差$T(t) - T_0$に比例し、
  \begin{equation}
    \frac{dT}{dt} = -k(T - T_0) \qquad (k > 0)\nonumber
  \end{equation}
  が成り立つ。\\
  この時、$100$度の銅球を$20$度の液体に入れた。\\
  ただし、液体の質量の温度は銅球の質量に比べて、十分に大きいものとする。\\
  $3$分後に銅球の温度は$80$度になった。この時、銅球が$21$度になるのは何分後か。\\

  \subsection*{解答}
  変数分離系の形なので、式変形を行うと、
  \begin{equation*}
    \frac{1}{T - T_0}dT = -kdt
  \end{equation*}
  が得られ、  初期条件である媒体の温度を代入し、両辺を積分すると、
  \begin{eqnarray*}
        \int \frac{1}{T - 20}dT & = & \int -kdt \\
        \ln |T - 20| & = & -kt + C \qquad (Cは任意定数)
  \end{eqnarray*}
  と積分できて、さらに式変形を行うと、\\
  \begin{eqnarray*}
        \ln |T - 20| & = & -kt + C \\
        \ln T - 20 & = & \pm (C - kt) \\
        T - 20 & = & \pm e ^ {C - kt} \\
        T(t) & = & 20 + C'e ^ {-kt} \qquad (C'は任意定数)
  \end{eqnarray*}
  $ t = 0 $ において、銅球の温度は$100$度のままであるからこれらを代入すると、
  \begin{eqnarray*}
    100 & = & 20 + C' \\
    C' & = & 80
  \end{eqnarray*}
  そして、$ t = 3 $において、
  \begin{eqnarray*}
    80 & = & 20 + 80 e ^ {-3k} \\
    e ^ {-3k} & = & \frac{60}{80} = \frac{3}{4}
  \end{eqnarray*}
  求める時刻での銅球の温度は $ 21 $ 度であるから、$ T(t) = 21 $ の時、
  \begin{eqnarray*}
    21 & = & 20 + 80e ^ {-kt}\\
    1 & = & 80 \left(e ^ {-3k} \right) ^ { \cfrac{t}{3}} \\
    1 & = & 80 \left( \frac{3}{4} \right) ^ {\frac{t}{3}} \\
    \frac{1}{80} & = & \left( \frac{3}{4} \right) ^ {\frac{t}{3}}
  \end{eqnarray*}
  が得られるので、両辺に対数を取ると、
  \begin{equation*}
    \frac{t}{3} \ln \left(\frac{3}{4}\right) = \ln \left(\frac{1}{80}\right)
  \end{equation*}
  したがって、
  \begin{equation*}
    t = 3 \cfrac{\ln \cfrac{1}{80}}{\ln \cfrac{3}{4}} = 3 \times \frac{-4.3820}{-0.2877} = 45.6966[min]
  \end{equation*}
  このようにして解が得られた。

  \begin{shadebox}
    $ \log $ という式を見ると、底を自然対数の $ e $ を思い浮かべる人が多いですが、この場合、普通は底は$ 10 $ で扱います。$ e $ のまま扱うのは数Ⅲくらいなのでこのことは常識として知っておきましょう。
    \newline
    また、本日の問題は物理の要素を交えた問題でした。この問題を完璧にしておくと、前期でやる微分方程式続論で楽になると思うのでしっかりと復習をしておきましょう。
  \end{shadebox}
  \end{qparts}
\end{spacing}
\end{document}
