\documentclass[dvipdfmx,uplatex]{jsarticle}
\def\vector#1{\mbox{\boldmath $#1$}}
\usepackage{qexam} % 問題を書く時とかに必要なやつ
\usepackage{setspace} % 行間開けるのに必要なやつ
\usepackage{amsmath} % 数学やるのに必要なやつ
\usepackage{bm} % 太字にするのに必要なやつ
\usepackage{amssymb}
\usepackage[hiresbb]{graphicx}
\usepackage{ascmac}
\usepackage{siunitx}
\usepackage{float}
\usepackage{tikz}
\usepackage{circuitikz}
\usepackage{url}
\usepackage{braket}
\usepackage[colorlinks=true, bookmarks=true,
bookmarksnumbered = true, bookmarkstype = toc, linkcolor = blue,
urlcolor=blue, citecolor=blue]{hyperref}
\usepackage[version=3]{mhchem}
\makeatletter
 \renewcommand{\theequation}{
   \thesubsection.\arabic{equation}}
  \@addtoreset{equation}{subsection}
\title{春休み毎日微分方程式 Day 5(解答)}
\author{公立はこだて未来大学 システム情報科学部 B2 日置竜輔}
\date{\today}

\begin{document}
\begin{spacing}{1.6}
\maketitle

\question{問1}
以下の計算をせよ。
\begin{qparts}
    \qpart $ \displaystyle (D ^ 2 + D)(x ^ 3 - 3x) $ \quad ただし、$ \displaystyle D = \frac{d}{dx} $ と定義する。\\
    微分演算子は普通の文字と同様の計算ができるので、展開して以下のように計算する。 \\
    \begin{eqnarray*}
      (D ^ 2 + D)(x ^ 3 - 3x) & = & D ^ 2 (x ^ 3 - 3x) + D (x ^ 3 - 3x)\\
      & = & \frac{d^2}{dx}(x ^ 3 - 3x) + \frac{d}{dx}(x ^ 3 - 3x)\\
      & = & (6x) + (3x ^ 2 - 3)\\
      & = & 3x ^ 2  + 6x - 3
    \end{eqnarray*}
    と計算できる。\\
    \begin{itembox}{微分演算子法の超基本事項}
      {\rm I} \quad $ D $ を重ね掛けして複数回の微分を表現可能 \\
      \begin{equation*}
        D^2 = \frac{d^2}{dx},\qquad D^3 = \frac{d ^ 3}{dx}, \qquad D ^ n = \frac{d^n}{dx}
      \end{equation*}
      {\rm II}\quad 定数は $ D $ の前に出せる。\\
      \begin{equation*}
        D^naf(x) = aD^nf(x)
      \end{equation*}
      {\rm III}\quad $ D $は分配・結合ができる。\\
      \begin{equation*}
        (D ^ m + D ^ n)f(x) = D^mf(x) + D^nf(x)
      \end{equation*}
    \end{itembox}
\end{qparts}
\end{spacing}

\end{document}
