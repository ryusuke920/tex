\documentclass[dvipdfmx,uplatex]{jsarticle}
\def\vector#1{\mbox{\boldmath $#1$}}
\usepackage{qexam} % 問題を書く時とかに必要なやつ
\usepackage{setspace} % 行間開けるのに必要なやつ
\usepackage{amsmath} % 数学やるのに必要なやつ
\usepackage{bm} % 太字にするのに必要なやつ
\usepackage{cases} % 連立方程式を書くのに必要なやつ
\usepackage{amssymb}
\usepackage[hiresbb]{graphicx}
\usepackage{ascmac}
\usepackage{siunitx}
\usepackage{float}
\usepackage{tikz}
\usepackage{circuitikz}
\usepackage{url}
\usepackage{braket}
\usepackage[colorlinks=true, bookmarks=true,
bookmarksnumbered = true, bookmarkstype = toc, linkcolor = blue,
urlcolor=blue, citecolor=blue]{hyperref}
\usepackage[version=3]{mhchem}
\makeatletter
 \renewcommand{\theequation}{
   \thesubsection.\arabic{equation}}
  \@addtoreset{equation}{subsection}
\title{春休み毎日微分方程式 Day 12(解答)}
\author{公立はこだて未来大学 システム情報科学部 B2 日置竜輔}
\date{2021年3月24日}

\begin{document}
\begin{spacing}{1.6}
\maketitle

\question{問1}
以下の微分方程式を解け。
\begin{qparts}
  \qpart
  \begin{equation*}
      (3y^2 - 2xy) + (x^2-2xy)y' = 0
  \end{equation*}
  \begin{eqnarray*}
    y' & = & \frac{-3^2 + 2xy}{x^2 - 2xy} \\
    & = & \frac{-3\left(\frac{y}{x}\right)^2 + 2\left(\frac{y}{x}\right)}{1 - 2\left(\frac{y}{x}\right)}
  \end{eqnarray*}
  $\displaystyle z = \frac{y}{x}$ と変数変換を行うと、 $y' = z + xz'$となるのでもとの微分方程式に代入すると、\\
  \begin{eqnarray*}
    z + xz' & = & \frac{-3z^2 + 2z}{1 - 2z} \\
    xz' & = & \frac{-3z^2 + 2z}{1 - 2z} - z \\
    & = & \frac{z - z^2}{1 - 2z}
  \end{eqnarray*}
  さらにここから変数分離を行うと、以下のように式変形できる。
  \begin{eqnarray*}
    \frac{1 - 2z}{z - z^2}dz & = & \frac{dx}{x} \\
    \int \frac{1 - 2z}{z - z^2} & = & \int \frac{dx}{x} \\
    \log |z - z^2| & = & log|x| + C' \\
    \log \left|\frac{z - z^2}{x}\right| & = & C' \\
    \frac{z - z^2}{x} & = & \pm e^{C'} \\
    z - z^2 & = & Cx
  \end{eqnarray*}
  $z$の式を$y$に直すと、
  \begin{eqnarray*}
    \frac{y}{x} - \frac{y^2}{x^2} & = & Cx \\
    xy - y^2 & = & Cx^3 \\
    y^2 - xy + Cx^3 & = & 0
  \end{eqnarray*}
  \end{qparts}
\end{spacing}
\begin{shadebox}
  今回はまた変数分離系と同次系の問題を出題しました。なかなか式変形などが大変だと思うので頑張って慣れましょう!
\end{shadebox}
\end{document}
