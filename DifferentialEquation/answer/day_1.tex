\documentclass[a4j]{jsarticle}
\def\vector#1{\mbox{\boldmath $#1$}}
\usepackage{qexam} % 問題を書く時とかに役に立つ
\usepackage{amsmath}
\usepackage{amssymb}
\usepackage[hiresbb]{graphicx}
\usepackage{ascmac}
\usepackage{siunitx}
\usepackage{float}
\usepackage{tikz}
\usepackage{circuitikz}
\usepackage{url}
\usepackage{braket}
\usepackage[colorlinks=true, bookmarks=true,
bookmarksnumbered = true, bookmarkstype = toc, linkcolor = blue,
urlcolor=blue, citecolor=blue]{hyperref}
\usepackage[version=3]{mhchem}
\makeatletter
 \renewcommand{\theequation}{
   \thesubsection.\arabic{equation}}
  \@addtoreset{equation}{subsection}
\title{春休み毎日微分方程式 Day 1(解答)}
\author{ryusuke\_h\thanks{Future University Hakodate B2}}
\date{2021年3月1日}

\begin{document}
\maketitle

\question{問1}
\begin{qparts}
    \qpart $ y ^ {\prime} = x $ の一般解を求めよ。
    両辺を $ x $ で積分すると、\\
    \begin{eqnarray}
      y & = & \int x dx \nonumber \\
      & = & \frac{1}{2} x ^ 2 + C \qquad (Cは任意定数)
    \end{eqnarray}
    となる。\\

    \qpart $ \rm I $ において、$ y(0) = 2 $ であるときの、微分方程式の特殊解を求めよ。\\
    (0.0.1)式において、$ y = 0 $ を代入すると、
    \begin{eqnarray}
      y(0) & = & \frac{1}{2} 0 ^ 2 + C = 2 \nonumber \\
      C & = & 2 \nonumber
    \end{eqnarray}
    となり、任意定数Cの値が求まるので特殊解は、
    \begin{align*}
      y = \frac{1}{2} x ^ 2 + 2
    \end{align*}
\end{qparts}

\end{document}
