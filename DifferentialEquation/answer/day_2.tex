\documentclass[a4j]{jsarticle}
\def\vector#1{\mbox{\boldmath $#1$}}
\usepackage{qexam} % 問題を書く時とかに必要なやつ
\usepackage{setspace} % 行間開けるのに必要なやつ
\usepackage{amsmath} % 数学やるのに必要なやつ
\usepackage{amssymb}
\usepackage[hiresbb]{graphicx}
\usepackage{ascmac}
\usepackage{siunitx}
\usepackage{float}
\usepackage{tikz}
\usepackage{circuitikz}
\usepackage{url}
\usepackage{braket}
\usepackage[colorlinks=true, bookmarks=true,
bookmarksnumbered = true, bookmarkstype = toc, linkcolor = blue,
urlcolor=blue, citecolor=blue]{hyperref}
\usepackage[version=3]{mhchem}
\makeatletter
 \renewcommand{\theequation}{
   \thesubsection.\arabic{equation}}
  \@addtoreset{equation}{subsection}
\title{春休み毎日微分方程式 Day 2(解答)}
\author{ryusuke\_h\thanks{Future University Hakodate B2}}
\date{2021年3月3日}

\begin{document}
\begin{spacing}{1.6}
\maketitle

\question{問1}
\begin{qparts}
    \qpart $ y ^ {\prime} = 2xy ^ 2 $ の一般解を求めよ。\\
    両辺を $ y ^ 2 $ で割ると、\\
    $ \displaystyle \int \cfrac{dy}{dx} \cfrac{1}{y ^ 2} = 2x $ \\
    両辺を $ x $ で積分すると、\\
    $ \displaystyle \int \frac{1}{y ^ 2}dy = \int 2xdx $ \\
    となるので、両辺を積分すると、\\
    $ \displaystyle -\frac{1}{y} = x ^ 2 + C \qquad (Cは任意定数) $ \\
    つまり、
    \begin{equation}
      y = -\frac{1}{x ^ 2 + C} \nonumber
    \end{equation}

    \qpart $\rm I $ において、$ y(0) = -1 $ となるような特殊解を求めよ。\\
    (0.0.1)において、$ x = 0 $ を 代入すると、$ \displaystyle -1 = \frac{1}{C}$ \\
    すなわち $ C = 1$ より、求める特殊解は、
    \begin{equation}
      y = -\frac{1}{x ^ 2 + 1} \nonumber
    \end{equation}
\end{qparts}
\end{spacing}

\end{document}
