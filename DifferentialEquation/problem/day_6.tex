\documentclass[a4j]{jsarticle}
\def\vector#1{\mbox{\boldmath $#1$}}
\usepackage{qexam} % 問題を書く時とかに必要なやつ
\usepackage{setspace} % 行間開けるのに必要なやつ
\usepackage{amsmath} % 数学やるのに必要なやつ
\usepackage{bm} % 太字にするのに必要なやつ
\usepackage{amssymb}
\usepackage[hiresbb]{graphicx}
\usepackage{ascmac}
\usepackage{siunitx}
\usepackage{float}
\usepackage{tikz}
\usepackage{circuitikz}
\usepackage{url}
\usepackage{braket}
\usepackage[colorlinks=true, bookmarks=true,
bookmarksnumbered = true, bookmarkstype = toc, linkcolor = blue,
urlcolor=blue, citecolor=blue]{hyperref}
\usepackage[version=3]{mhchem}
\makeatletter
 \renewcommand{\theequation}{
   \thesubsection.\arabic{equation}}
  \@addtoreset{equation}{subsection}
\title{春休み毎日微分方程式 Day 6(問題)}
\author{ryusuke\_h\thanks{Future University Hakodate B2}}
\date{2021年3月11日}

\begin{document}
\begin{spacing}{1.6}
\maketitle

\question{問1}
以下の証明をせよ。 ($ \displaystyle D = \frac{d}{dx} $ を表す。)
\begin{qparts}
    \qpart {\bm 微分演算子} と {\bm 定数変化法}を用いて、
    $ \displaystyle \frac{dy}{dx} + ay = f(x) から $
    \begin{equation}
      \frac{1}{D + a}f(x) = e ^ {-ax} \left(\int e ^ {ax} f(x) dx \right )
    \end{equation}を導出せよ。 \\
  \end{qparts}
\end{spacing}

\end{document}
