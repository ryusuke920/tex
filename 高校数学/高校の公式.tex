\documentclass[dvipdfmx,uplatex]{jsarticle}
\def\vector#1{\mbox{\boldmath $#1$}}
\usepackage{amsmath}
\usepackage{amssymb}
\usepackage[hiresbb]{graphicx}
\usepackage{ascmac}
\usepackage{siunitx}
\usepackage{float}
\usepackage{tikz}
\usepackage{circuitikz}
\usepackage{url}
\usepackage{braket}
\usepackage[colorlinks=true, bookmarks=true,
bookmarksnumbered = true, bookmarkstype = toc, linkcolor = blue,
urlcolor=blue, citecolor=blue]{hyperref}
\usepackage[version=3]{mhchem}
\makeatletter
 \renewcommand{\theequation}{
   \thesubsection.\arabic{equation}}
  \@addtoreset{equation}{subsection}
\title{覚えるべき公式(高校編)}
\author{公立はこだて未来大学 システム情報科学部 B2 日置竜輔}
\date{\today}

\begin{document}
\maketitle

\section{三角関数}

\begin{eqnarray}
  \cfrac{\sin x}{\cos x} & = & \tan x \\
  \sin x ^ 2 + \cos x ^ 2 & = & 1 \\
  1 + \tan ^ 2 x & = & \cfrac{1}{\cos ^ 2 x}
\end{eqnarray}

\subsection{公式の導出について}

(1.0.3)については、(1.0.2)の両辺に $ \cfrac{1}{\cos ^ 2 x} $ をかけると、
\begin{equation}
  \cfrac{\sin ^ 2 x}{\cos ^ 2 x} + \cfrac{\cos ^ 2 x}{\cos ^ 2 x} = \cfrac{1}{\cos ^ 2 x} \nonumber
\end{equation}

すなわち、

\begin{equation}
  1 + \tan ^ 2 x = \cfrac{1}{\cos ^ 2 x} \nonumber
\end{equation}

が得られる。

\end{document}
