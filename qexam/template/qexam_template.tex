\documentclass[dvipdfmx,11pt,a4j]{jarticle}  
% qexam_template.tex - 試験問題のひな形
% 2009-07-11 Taku Yamanaka (Physics Dept., Osaka Univ.)
% 2019-06-20 Taku: Updated for v2.0.

%-------- page style ----------------------
%\usepackage[a4paper]{geometry}
%\usepackage[a4paper,text={17cm,25cm},centering]{geometry}	% to expand the text area

%\pagestyle{empty}		% suppresses page numbering

%--------- users packages ---------------
%% AMS math package
%\usepackage{amsmath}
%% eps ファイルを読み込むためのパッケージ
\usepackage{graphicx}
%\usepackage{bm} % for vectors in bold math
%\usepackage{udline} % for underlines crossing multiple lines

\usepackage{qexam} %試験問題用パッケージ

%====================================
\begin{document}
%% 和文フォントをゴシック体に変更する。
%% 明朝体でいいときは、コメントアウトする。
%\fontfamily{gt}
%\selectfont
%%%%

%\renewcommand{\questionFormat}[1]{%
%% Here are some examples.  Tune them as you like.
%%	\textbf{\Large{#1}}
%%	\framebox{\LARGE{#1}}
%%	\hspace{-5mm}\textbf{\Huge{[#1]}}
%%	\begin{center}{\textbf{\LARGE{#1}}}\end{center}
%}

%% 問1用のヘディング%%%%%%%%%%%%%%%%%%%%%%%%%%%%%%%%
\begin{center}
	\textbf{\Large  qexam.sty による\\
	\bigskip
	物理学 試験問題}\\
	\bigskip
	\textbf{\large 2999年9月9日}
\end{center}

問題1から問題2までのすべての問題に解答せよ。
解答用紙は問題ごとに1枚とし,それぞれに氏名・受験番号・問題番号を書くこと。
%%%%%%%%%%%%%%%%%%%%%%%%%%%%%%%%%%%%%%%%%%%%%

\question{問1}
地の文。
\begin{qlist}
	\qitem 小問
	\qitem 小問
\end{qlist}

\begin{qparts}
	\qpart パート1
		\begin{qlist}
			\qitem 小問
			\qitem 小問
		\end{qlist}
	\qpart パート2
		\begin{qlist}
			\qitem 小問
			\qitem 次の \qbox{(a) -- (b)}を埋めよ。\\
				この \qbox{}埋め 問題に箱は \qbox{}個ある。
		\end{qlist}
\end{qparts}

\end{document}

